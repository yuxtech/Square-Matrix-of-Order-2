\chapter{Cayley--Hamilton定理}
\begin{proverb}
  { \kaishu
      If $A\in\MM_2(\MC)$, then $A^2-\Tr(A)A + (\det A)I_2=O_2.$
  }

\hfill Cayley--Hamilton
\end{proverb}

\section{Cayley--Hamilton定理}

如果$A=\begin{pmatrix}
  a & b \\
  c & d
\end{pmatrix}\in\MM_2(\MC)$,则:
\begin{itemize}
  \item $A$的{\kaishu 特征多项式}\index{T!特征多项式}定义为
      \[
        f_A(x) = \det (A - xI_2) = x^2 - (a+d)x + ad -bc = x^2 - \Tr(A)x + \det A \in \MC[x];
      \]
  \item 方程
      \[
        f_A(\lambda) = 0 \Leftrightarrow
        \lambda^2 - \Tr(A)\lambda + \det A = 0
      \]
      称为$A$的{\kaishu 特征方程};\index{T!特征方程}
  \item 特征方程的根$\lambda_1,\lambda_2$称为$A$的{\kaishu 特征值}\index{T!特征值},且集合$\{\lambda_1,\lambda_2\}$称为$A$的{\kaishu 谱}\index{P!谱},记作$\Spec(A)$.
\end{itemize}

由Vi\`ete公式,\index{V!Vi\`ete公式}有
\begin{mybox}
  \hfil
    $\lambda_1 + \lambda_2 =\Tr(A)$\quad 且
    \quad $\lambda_1\lambda_2 = \det A$.
  \hfill
\end{mybox}

接下来,我们给出一些矩阵的特征值的性质,这些性质可以通过直接计算得到.

\begin{mybox}
  设$\lambda\in\MC$,则
  \begin{itemize}
    \item 如果$\lambda$不是$A$的特征值,则方程组
      \[
        \left\{
          \begin{aligned}
            & (a-\lambda)x + by = 0 \\
            & cx + (d-\lambda)y = 0
          \end{aligned}
        \right.\quad \Leftrightarrow\quad
        AX = \lambda X,\,\text{其中}\,X = \begin{pmatrix}
          x \\ y
        \end{pmatrix},
      \]
      只有平凡解$(x,y)=(0,0)$.
    \item 如果$\lambda$是$A$的特征值,则方程组$AX=\lambda X$至少有一个非平凡解$X\ne0$;这样一个解称为$A$的相应于特征值$\lambda$的一个{\kaishu 特征向量}.\index{T!特征向量}
    \item $\lambda \in\MC$是$A$的一个特征值当且仅当存在
        \[
          X = \begin{pmatrix}
            x \\ y
          \end{pmatrix}\quad \text{使得} \quad AX=\lambda X;
        \]
  \end{itemize}

    如果$\lambda_1,\lambda_2$是$A$的特征值,则
    \begin{itemize}
      \item $\lambda_1^n,\lambda_2^n$是$A^n,n\in\MN$的特征值;
      \item 对任意多项式$P\in\MC[x]$,$P(\lambda_1),P(\lambda_2)$是矩阵$P(A)$的特征值;
      \item 如果$A$可逆,$\det A=\det(A-0I_2)\ne0$,所以0不是$A$的特征值,则$\frac1{\lambda_1},\frac1{\lambda_2}$是$A^{-1}$的特征值.
    \end{itemize}
\end{mybox}

接下来的定理给出两个可交换矩阵的和与乘积的特征值.

\begin{mybox}
  \begin{theorem}[两个可交换矩阵的和与乘积的特征值.]

    如果$A,B\in\MM_2(\MC)$是可交换的矩阵,则矩阵$A+B$和$AB$的特征值形式为
    \[
      \lambda_{A+B} = \lambda_A + \lambda _B
      \quad \text{且} \quad \lambda_{AB} = \lambda_A\lambda_B.
    \]
  \end{theorem}
\end{mybox}

\begin{proof}
  如果$B=\alpha I_2$,则$A+B=A+\alpha I_2$的特征值为$\lambda_1+\alpha$和$\lambda_2+\alpha$,其中$\lambda_1,\lambda_2$是$A$的特征值,而$\alpha$是$B$的特征值. 另一方面,$AB=\alpha A$的特征值的特征值为$\alpha\lambda_1$和$\alpha\lambda_2$.

  如果$B\ne\alpha I_2,\alpha\in\MC$,则$B\in\mathscr C (A)$,且由定理 \ref{thm1.1} 的 \ref{thm1.1itemb} 我们有$B=\alpha I_2+\beta A$对某个$\alpha,\beta\in\MC$成立. 我们有
  \[
    \lambda_B = \alpha + \beta\lambda_A,\,
    A + B = \alpha I_2 + (\beta+1)A,\,
    AB = \alpha A + \beta A^2.
  \]
  由此得到$\lambda_{A+B}=\alpha+(\beta+1)\lambda_A=\lambda_A+\lambda_B$,
  且$\lambda_{AB}=\alpha\lambda_A+\beta\lambda_A^2
  =\lambda_A\lambda_B $.
\end{proof}

\begin{remark}
  如果$\alpha,\beta\in\MC$且$A,B\in\MM_2(\MC)$是可交换的矩阵,则矩阵$\alpha A+\beta B$和$A^iB^k$的特征值形式为
  \[
    \lambda_{\alpha A+\beta B} = \alpha\lambda_A + \beta\lambda_B\quad \text{且}\quad \lambda_{A^iB^k}=\lambda_A^i\lambda_B^k.
  \]
\end{remark}

现在我们就可以着手证明著名的Cayley--Hamilton定理了,这个定理指出{\kaishu 任意方阵的特征多项式是这个方阵的零化多项式}.

\begin{mybox}
  \begin{theorem}[Cayley--Hamilton定理]

    如果$A\in\MM_2(\MC)$,则$A^2-\Tr(A)A+(\det A)I_2=O_2$.
  \end{theorem}
\end{mybox}

\begin{proof}
  我们通过直接计算来证明这个定理. 设$A=\begin{pmatrix}
    a & b \\
    c & d
  \end{pmatrix}$.

  直接计算可得
  \[
    A^2 = \begin{pmatrix}
      a^2 + bc & b(a + d) \\
      c(a+d) & d^2 + bc
    \end{pmatrix}.
  \]

  如果$x=\Tr(A)=a+d$,则
  \begin{align*}
    A^2 - \Tr(A) + (\det A)I_2 = {} &\begin{pmatrix}
      a^2 + bc & b(a + d) \\
      c(a+d) & d^2 + bc
    \end{pmatrix} -
    \begin{pmatrix}
      ax & bx \\
      cx & dx
    \end{pmatrix} \\
    & + \begin{pmatrix}
      ad - bc & 0 \\
      0 & ad - bc
    \end{pmatrix} \\
    = {}& \begin{pmatrix}
      a^2 + ad - ax & 0 \\
      0 & d^2 + ad - dx
    \end{pmatrix}\\
    = {}& O_2,
  \end{align*}
  这就证明了定理.
\end{proof}

{\kaishu 历史注记.} Cayley--Hamilton定理是在1853年由Hamilton利用四元数的线性函数所证明 \cite{36}. 这相应于$4\times4$的实矩阵或者$2\times2$的复矩阵的特殊情形. 在1958年,Cayley指出对$3\times3$矩阵也成立,但只发表了$2\times2$矩阵的情形 \cite{14}. “一般来说,Cayley不是一个容易激动的人,在发现之时,他宣称Cayley--Hamilton定理“非常了不起”,几代数学家都分享了他的喜悦”\cite[p.772]{35}. 然而,是Frobenius在1878年证明了一般的情形 \cite{19}.

接下来我们给出Cayley--Hamilton定理的一些应用.

\begin{lemma}
  如果$A\in\MM_2(\MC)$是可逆的,则
  \[
    A^{-1} = \frac1{\det A} \big( \Tr(A)I_2 - A \big)\quad \text{且}\quad \Tr(A^{-1}) =
    \frac{\Tr(A)}{\det A}.
  \]
\end{lemma}

\begin{proof}
  由Cayley--Hamilton定理,我们有$A^2-\Tr(A)A+(\det A)I_2=O_2$. 将这个等式两边乘以$A^{-1}$,我们得到
  \[
    A - \Tr(A)I_2 + (\det A)A^{-1} = O_2 \Rightarrow A^{-1} = \frac1{\det A}\big(
    \Tr(A)I_2 - A\big).
  \]
  引理的第二部分只需要在第一部分中取迹即可.
\end{proof}

接下来的引理是关于计算行列式为0或者迹为0的二阶矩阵的幂.

\begin{lemma}[两类特殊矩阵的$n$次幂]
  \begin{enum}
    \item 如果$A\in\MM_2(\MC)$满足$\det A=0$,则
      \[
        A^n = \big( \Tr(A) \big)^{n-1}A,\,\forall n\in\MN.
      \]
    \item 如果$A\in\MM_2(\MC)$满足$\Tr(A)=0$,则
      \[
        A^n = \begin{cases}
          (-\det A)^kI_2,  & n = 2k,k\in\MN \\
          (-\det A)^{k-1}A & n = 2k-1, k\in\MN
        \end{cases}.
      \]
  \end{enum}
\end{lemma}

\begin{proof}
  \begin{enuma}
    \item 由定理 \ref{thm2.2},我们有$A^2=\Tr(A)A$,这意味着
  \end{enuma}
    \[
      A^3 = A^2A =(\Tr A)AA = \Tr(A)\Tr(A)A = \Tr^2(A)A.
    \]
  由数学归纳法我们有$A^n=(\Tr A)^{n-1}A,\,\forall n\in\MN$.

  \begin{enuma}
    \setcounter{enumi}{1}
    \item 由于$\Tr(A)=0$,利用定理 \ref{thm2.2},我们得到$A^2+(\det A)I_2=O_2$. 这意味着$A^2=-(\det A)I_2$,那么当$n$是奇数或者偶数的情形,利用数学归纳法就可以证明了.
  \end{enuma}
\end{proof}

\begin{lemma}
  设$A\in\MM_2(\MC)$,则下列叙述是等价的:
  \begin{enum}
    \item \label{lemma2.3.a} $A^2=O_2$;
    \item \label{lemma2.3.b} 存在$n\in\MN,n\ge2$使得$A^n=O_2$.
  \end{enum}
\end{lemma}

\begin{proof}
  \ref{lemma2.3.a} $\Rightarrow$ \ref{lemma2.3.b} 是显然的. 要证明 \ref{lemma2.3.b} $\Rightarrow$ \ref{lemma2.3.a},我们注意到$A$的特征值均为0,因此$A$的特征多项式为$f_A(x)=x^2$. 由定理 \ref{thm2.2} 可得$A^2=O_2$,引理得证.
\end{proof}

作为引理 \ref{lemma2.3} 的推论,我们有如果$A^2\ne O_2$,则$A$的任意次幂都非零. 我们把它表述为下面的引理:

\begin{lemma}
  如果$A\in\MM_2(\MC)$满足$A^2\ne O_2$,则$A^n\ne O_2$对任意$n\in\MN$成立.
\end{lemma}

\begin{lemma}[幂零矩阵的一个事实.]

设$A,B\in\MM_2(\MC)$,如果$A$和$B$都是幂零矩阵,且$AB=BA$,则$A+B$和$A-B$都是幂零矩阵.
\end{lemma}

\begin{proof}
  要证明引理等价于证明,如果$A$和$B$是可交换的矩阵,且$A^2=B^2=O_2$,则$(A\pm B)^2=O_2$. 我们有$(A\pm B)^2=A^2\pm2AB+B^2=\pm2AB$,而这意味着$(A\pm B)^2=4A^2B^2=O_2$. 再结合引理 \ref{lemma2.3} 就能得出结果.
\end{proof}

现在我们将注意力转向定理 \ref{thm2.2} 的有关二阶矩阵行列式的应用.

\begin{lemma}[用迹来表示行列式.]

  如果$A\in\MM_2(\MC)$,则
  \begin{equation}\label{eq2.1}
    \det A = \frac12\big[ (\Tr A)^2 -\Tr(A^2) \big].
  \end{equation}
\end{lemma}

\begin{proof}
  由定理 \ref{thm2.2},我们有$A^2-\Tr(A)A+(\det A)I_2=O_2$. 等式两边取迹,我们得到
  \[
    \Tr(A^2) - \Tr(A)\Tr(A) + 2\det A = 0,
  \]
  引理得证.
\end{proof}

\begin{remark}
  Cayley--Hamilton定理的另一种形式:由公式 \eqref{eq2.1},我们有下面的公式
  \[
    A^2 - \Tr(A)A + \frac12\big[ (\Tr A)^2 -\Tr(A^2) \big]I_2 = O_2,\,\forall A\in\MM_2(\MC).
  \]
\end{remark}

\begin{mybox}
  \begin{lemma}[\cite{45} 一个重要的行列式公式]

    如果$A,B\in\MM_2(\MC)$且$x\in\MC$,则
    \begin{equation}\label{eq2.2}
      \det (A + xB) = \det A + \big(
        \Tr(A)\Tr(B) - \Tr(AB)
      \big)x + (\det B)x^2.
    \end{equation}
  \end{lemma}
\end{mybox}

\begin{proof}
  由公式 \eqref{eq2.1},我们有
  \begin{align*}
    \det(A + xB) = {}& \frac12\left[ \big( \Tr(A + xB) \big)^2 - \Tr\big( (A + xB)^2 \big) \right]\\
    = {}& \frac12\big[ \big(\Tr(A)+x\Tr(B)\big)^2 - \Tr(A^2+xAB+xBA+B^2x^2) \big] \\
    = {}& \frac12\big[ (\Tr A)^2 + 2\Tr(A)\Tr(B)x +
     (\Tr B)^2x^2 - \Tr(A^2)\\
     & - 2\Tr(AB)x - \Tr(B^2)x^2 \big] \\
    = {}& \frac12\big[ (\Tr A)^2 - \Tr(A^2) \big]
      + \big( \Tr(A)\Tr(B) - \Tr(AB) \big)x \\
      & + \frac12\big[ (\Tr B)^2 - \Tr(B^2) \big]x^2 \\
    = {}& \det A + \big(
        \Tr(A)\Tr(B) - \Tr(AB)
      \big)x + (\det B)x^2,
  \end{align*}
  因此引理得证.
\end{proof}

\begin{corollary}
  如果$A,B\in\MM_2(\MC)$,则
  \[
    \det (A + B) + \det(A - B) = 2\det A + 2\det B.
  \]
\end{corollary}
\begin{proof}
  在公式 \eqref{eq2.2} 中分别取$x=1$和$x=-1$,然后将两个等式相加即得. 也可以参见问题 \ref{problem1.31} 的 \ref{prob1.31.a} 得到另一种证法.
\end{proof}

\begin{mybox}
  \begin{corollary}[行列式和迹的性质.]

    如果$A,B\in\MM_2(\MC)$,则:
    \begin{enum}
      \item\label{coro.a} $\det(A+B)-\det A-\det B=\Tr(A)\Tr(B)-\Tr(AB)$;
      \item\label{coro.b} $\det(A-B)-\det A-\det B=\Tr(AB)-\Tr(A)\Tr(B)$;
      \item\label{coro.c} $\det(A+B)-\det(A-B)=2\big(
      \Tr(A)\Tr(B)-\Tr(AB)\big)$.
    \end{enum}
  \end{corollary}
\end{mybox}

\begin{proof}
  \ref{coro.a} 和 \ref{coro.b} 可以分别在公式 \eqref{eq2.2} 中取$x=1$和$x=-1$得到,而 \ref{coro.c} 可以通过在 \ref{coro.a} 中减去 \ref{coro.b} 得到.
\end{proof}

\begin{mybox}
  \begin{theorem}[极化Cayley--Hamilton定理.]

    如果$A,B\in\MM_2(\MC)$,则
    \[
      AB + BA - \Tr(A)B - \Tr(B)A + [
      \Tr(A)\Tr(B) - \Tr(AB)]I_2 = O_2.
    \]
  \end{theorem}
\end{mybox}

\begin{proof}
  设$x\in\MR$. 对矩阵$A+xB$应用Cayley--Hamilton定理,我们有
  \[
    (A + xB)^2 - \Tr(A + xB)(A + xB) + \det(A + xB)I_2 = O_2.
  \]
  由于$(A+xB)^2=A^2+B^2x^2+x(AB+BA)$,利用引理 \ref{lemma2.7} 我们有
  \begin{align*}
    A^2 + B^2x^2 & + (AB + BA)x - \big[
      \Tr(A)A + x\big( \Tr(B)A + \Tr(A)B \big) + x^2\Tr(B)B
    \big]\\
    & + \big[
      \det A + \big( \Tr(A)\Tr(B) - \Tr(AB)\big)x
      + x^2\det B
    \big]I_2 = O_2.
  \end{align*}
  在上述等式中令$x=1$,定理得证.
\end{proof}

\begin{mybox}
  \begin{corollary}
    设$A,B,C\in\MM_2(\MC)$,则:
    \begin{enum}
      \item\label{coro2.3a} Cayley--Hamilton定理的另一种极化形式
      \begin{align*}
        2ABC = {}& \Tr(A)BC + \Tr(B)AC + \Tr(C)AB - \Tr(AC) B \\
        & + [\Tr(AB) - \Tr(A)\Tr(B)]C + [\Tr(BC) - \Tr(B)\Tr(C)]A \\
        & - [\Tr(ACB) - \Tr(AC)\Tr(B)]I_2;
      \end{align*}
      \item 迹的一个性质
      \begin{align*}
        \Tr(ABC) = {}& \Tr(A)\Tr(BC) + \Tr(B)\Tr(AC) + \Tr(C)\Tr(AB) \\
        & - \Tr(ACB) - \Tr(A)\Tr(B)\Tr(C).
      \end{align*}
    \end{enum}
  \end{corollary}
\end{mybox}

\begin{proof}
  \begin{enuma}
    \item 由定理 \ref{thm2.3},我们有
  \end{enuma}
  \begin{align*}
    2ABC = {}& A(BC + CB) + (AB + BA)C - [B(AC) + (AC)B] \\
    = {}& A \big[ \Tr(B)C + \Tr(C)B - \big( \Tr(B)\Tr(C) -\Tr(BC) \big)I_2 \big]\\
        & + \big[ \Tr(A)B + \Tr(B)A - \big( \Tr(A)\Tr(B) -\Tr(AB) \big)I_2 \big]C \\
        & - \big[ \Tr(AC)B + \Tr(B)AC - \big( \Tr(B)\Tr(AC) -\Tr(ACB) \big)I_2 \big] \\
    = {}& \Tr(A)BC + \Tr(B)AC + \Tr(C)AB - \Tr(AC)B \\
        & + \big[
          \Tr(AB) - \Tr(A)\Tr(B)
        \big]C + \big[ \Tr(BC) - \Tr(B)\Tr(C) \big] A \\
        & - [\Tr(ACB) - \Tr(AC)\Tr(B)]I_2.
  \end{align*}

  \begin{enuma}
    \setcounter{enumi}{1}
    \item 这一部分的性质,只需要将 \ref{coro2.3a} 中的等式取迹即可得到.
  \end{enuma}
\end{proof}

\begin{mybox}
  \begin{corollary}[一个特殊系数的多项式.]

    如果$A,B\in\MM_2(\MC)$且$x,y\in\MC$,则
    \[
      \det (xA + yB) = x^2\det A + y^2\det B + xy[ \det (A + B) - \det A - \det B].
    \]
  \end{corollary}
\end{mybox}

\begin{proof}
  如果$x=0$,等式显然成立. 如果$x\ne0$,令$\alpha=\frac yx$,我们有
  \begin{align*}
    \det (xA + yB) & = x^2 \det (A + \alpha B )\\
    & = x^2\big[ \det A + \big( \Tr(A)\Tr(B) - \Tr(AB)\big)\alpha +\alpha^2\det B  \big] \\
    & = x^2 \det A + xy[\Tr(A)\Tr(B) - \Tr(AB)] + y^2\det B \\
    & = x^2 \det A + xy[\det (A + B) - \det A - \det B] + y^2\det B,
  \end{align*}
  其中第二个等号由引理 \ref{lemma2.7}, 最后的等号由推论 \ref{coro2.2} 得到.
\end{proof}

\begin{corollary}
  如果$A\in\MM_2(\MC)$且$x\in\MC$,则$\det (A+xI_2)=\det A+\Tr(A)x+x^2$.
\end{corollary}

\begin{proof}
  这由引理 \ref{lemma2.7} 中取$B=I_2$即证.
\end{proof}

\begin{lemma}
  如果$A\in\MM_2(\MC)$,则$A+A_\ast=\Tr(A)I_2$.
\end{lemma}

\begin{proof}
  这个引理可以直接计算证明.
\end{proof}

\begin{lemma}
  如果$A,B\in\MM_2(\MC)$,则
  \[
    \Tr(A_\ast B) = \Tr(AB_\ast) = \Tr(A)\Tr(B) - \Tr(AB).
  \]
\end{lemma}

\begin{proof}
  由引理 \ref{lemma2.8},我们有
  \[
    A_\ast = \Tr(A)I_2 - A \Rightarrow
    \Tr(A_\ast B) = \Tr\big( \Tr(A)B - AB \big)
    = \Tr(A)\Tr(B) - \Tr(AB).
  \]
  类似地,也有$\Tr(AB_\ast) = \Tr(A)\Tr(B)-\Tr(AB)$.
\end{proof}

接下来的推论是引理 \ref{lemma2.7} 和 引理 \ref{lemma2.9} 的结果.

\begin{corollary}
  如果$A,B\in\MM_2(\MC)$且$x\in\MC$,则
  \[
    \det (A + xB) = \det A + \Tr(AB_\ast)x + (\det B)x^2.
  \]
\end{corollary}

\begin{lemma}
  如果$A,B\in\MM_2(\MC)$,则$\det(AB-BA)=\Tr(A^2B^2)
  -\Tr\big((AB)^2\big) $.
\end{lemma}

\begin{proof}
  由公式 \eqref{eq2.2},我们有
  \[
    \det (AB - BA) = \det (AB) - \big( \Tr(AB)\Tr(BA) - \Tr(AB^2A) \big) + \det (BA).
  \]

  我们注意到$\det(AB)=\det(BA),\Tr(AB)=\Tr(BA),
  \Tr(AB^2A)=\Tr(A^2B^2) $, 由此得到
  \[
    \det (AB - BA) = 2\det (AB) - \big(\Tr(AB)\big)^2 + \Tr(A^2B^2),
  \]
  再结合公式 \eqref{eq2.1} 就证明了引理.
\end{proof}

\begin{lemma}
  如果$A,B\in\MM_2(\MC)$,则
  \[
    \det (A - B) \det (A + B) = \det (A^2 - B^2) + \det (AB - BA).
  \]
\end{lemma}

\begin{proof}
  由推论 \ref{coro2.1},我们有
  \begin{align*}
    \det[(A^2 - B^2) + (AB - BA)] + &\det [(A^2 - B^2) - (AB - BA)] \\
    & = 2\big[ \det(A^2 - B^2) + \det(AB - BA) \big].
  \end{align*}
  然而
  \[
    \det [ (A^2 - B^2) + (AB - BA) ] = \det [ (A - B)(A + B) ] = \det (A - B) \det (A + B),
  \]
  且
  \[
    \det [ (A^2 - B^2) - (AB - BA) ] = \det [(A + B) (A - B)]  = \det (A - B) \det (A + B),
  \]
  于是引理得证.
\end{proof}

\begin{lemma}
  如果$A,B\in\MM_2(\MC)$,则
  \[
    \det (A^2 + B^2) = \det (AB - BA) + (\det A - \det B)^2 + [ \det (A + B) - \det A - \det B ]^2.
  \]
\end{lemma}

\begin{proof}
  我们应用引理 \ref{lemma2.11},将$B$换成$\ii B$得到
  \[
    \det (A - \ii B) \det (A + \ii B) = \det (A^2 + B^2) + \det [\ii (AB - BA)].
  \]
  这意味着
  \begin{equation}\label{eq2.3}
    \det (A^2 + B^2) = \det (AB - BA) + \det (A - \ii B) \det (A + \ii B).
  \end{equation}

  另一方面,由公式 \eqref{eq2.2} 我们有
  \[
    \det (A + \ii B) = \det A - \det B + [\Tr(A)\Tr(B) - \Tr(AB)] \ii,
  \]
  且
  \[
    \det (A - \ii B) = \det A - \det B - [\Tr(A)\Tr(B) - \Tr(AB)] \ii.
  \]
  由此得到
  \[
    \det (A - \ii B) \det (A + \ii B) = (\det A - \det B)^2 + [\Tr(A)\Tr(B) - \Tr(AB)]^2.
  \]
  再结合推论 \ref{coro2.2} 的 \ref{coro.a} 就得到
  \begin{equation}\label{eq2.4}
    \det (A - \ii B) \det (A + \ii B) = (\det A - \det B)^2 + [\det (A + B) - \det A - \det B]^2.
  \end{equation}
  结合 \eqref{eq2.3} 和 \eqref{eq2.4} 就证明了引理.

\end{proof}

\begin{mybox}
  \begin{theorem}[矩阵幂等式]

    设$\lambda_1,\lambda_2$是$A\in\MM_2(\MC)$的特征值,则成立下面的性质:
    \begin{enum}
      \item \label{thm2.4.a} $(A-\lambda_1I_2)^{2n}+(A-\lambda_2I_2)^{2n}
          =(\lambda_2-\lambda_1)^{2n}I_2,\,n\ge1$;
      \item \label{thm2.4.b} $(A-\lambda_1I_2)^{2n-1}-(A-\lambda_2I_2)^{2n-1}
          =(\lambda_2-\lambda_1)^{2n-1}I_2,\,n\ge1$
    \end{enum}
  \end{theorem}
\end{mybox}

\begin{proof}
  \begin{enuma}
    \item 我们通过对$n$归纳证明定理的 \ref{thm2.4.a} 部分. 设$P(n)$表示所要证明的问题
  \end{enuma}
    \[
      P(n):\quad (A-\lambda_1I_2)^{2n}+(A-\lambda_2I_2)^{2n}
          =(\lambda_2-\lambda_1)^{2n}I_2.
    \]

    首先,我们证明$P(1)$成立,即需要证明等式
    \[
      (A - \lambda_1I_2)^2 + (A - \lambda_2I_2)^2 = (\lambda_2 - \lambda_1)^2I_2
    \]
    成立. 我们有
    \begin{align*}
      (A - \lambda_1I_2)^2 + (A - \lambda_2I_2)^2  & = A^2 - 2\lambda_1 A +\lambda_1^2I_2 + A^2 - 2\lambda_2A + \lambda_2^2I_2 \\
      & = 2[A^2 - (\lambda_1 + \lambda_2)A + \lambda_1\lambda_2I_2] + (\lambda_1^2 - 2\lambda_1\lambda_2 + \lambda_2^2)I_2 \\
      & = (\lambda_2 - \lambda_1)^2 I_2,
    \end{align*}
    其中最后一步等号是由Cayley--Hamilton定理得到.

    现在我们假定$P(k)$对$k=1,2,\cdots,n$成立,我们要证明$P(n+1)$也成立. 由于$P(1)$和$P(n)$成立,我们有
    \begin{align*}
      & (A-\lambda_1I_2)^{2n+2}+(A-\lambda_2I_2)^{2n+2}\\
      = {} & \big[ (A - \lambda_1I_2)^{2n} + (A - \lambda_2I_2)^{2n} \big]\big[ (A - \lambda_1I_2)^2 + (A - \lambda_2I_2)^2 \big]\\
      & - (A - \lambda_1I_2)^{2n}(A - \lambda_2I_2)^2 - (A - \lambda_2I_2)^{2n}(A - \lambda_1I_2)^2 \\
      = {} & (\lambda_2 - \lambda_1)^{2n}(\lambda_2 -\lambda_1)^2I_2 \\
      = {}& (\lambda_2 - \lambda_1)^{2n+2}I_2,
    \end{align*}
    其中我们用到了$(A-\lambda_1I_2)(A-\lambda_1I_2)
    =(A-\lambda_2I_2)(A-\lambda_1I_2)=O_2$.

    \begin{enuma}
      \setcounter{enumi}{1}
      \item 当$n=1$时是显然的. 令$X=A-\lambda_1I_2$且$Y=A-\lambda_2I_2$,我们有
    \end{enuma}
      \begin{align*}
        X^{2n-1} - Y^{2n-1} & = (X^{2n-2} + Y^{2n-2})(X - Y) + X^{2n-2}Y - Y^{2n-2}X \\
        & \overset{\ref{thm2.4.a}}{=} (\lambda_2 - \lambda_1)^{2n-2} (\lambda_2 - \lambda_1)I_2 \\
        & = (\lambda_2 - \lambda_1)^{2n-1}I_2,
      \end{align*}
    其中我们利用了$XY=YX=O_2$. 证毕.
\end{proof}

\section{对称矩阵的特征值}

在本节中,我们证明实的$2\times2$矩阵是可对角化的,并且将$A$对角化的矩阵$P\in\MM_2(\MR)$可以取为一个正交矩阵,即$P\TT=P^{-1}$. 事实上,$P$是一个旋转矩阵. 这种思想常用在第 \ref{chap4} 章 和第 \ref{chap6} 章 中不同区域上二重积分的计算,也常用于第 \ref{chap6} 章中将一个二次曲线化为它的标准形.

\begin{mybox}
  \begin{theorem}[对称矩阵及其特征值.]

  设$A=\begin{pmatrix}
    a & b \\
    b & d
  \end{pmatrix}\in\MM_2(\MR)$.
  \begin{enum}
    \item $A$有实特征值
      \[
        \lambda_1 = \frac{a + d + \sqrt{(a - d)^2 + 4b^2} }2 ,\,\lambda_2 = \frac{a + d - \sqrt{(a - d)^2 + 4b^2 } }2,
      \]
      且$\lambda_1=\lambda_2=\lambda$当且仅当$A=\lambda I_2$.
    \item $A$可对角化,并且将$A$对角化的矩阵$P\in\MM_2(\MR)$可以取为一个正交矩阵. 我们有
        \[
          P^{-1}AP = \begin{pmatrix}
            \lambda_1 & 0 \\
            0 & \lambda_2
          \end{pmatrix},\;\text{其中}\, P=R_\theta,\,\tan\theta =
          \frac{d - a + \sqrt{(a-d)^2+4b^2}}{2b},b\ne0.
        \]
  \end{enum}
  \end{theorem}
\end{mybox}

\begin{proof}
  \begin{enuma}
    \item 直接计算即可证明.
    \item 可逆矩阵$P$的列是分别相应于特征值$\lambda_1$和$\lambda_2$的特征向量. 如果$v_i,i=1,2$是相应于特征值$\lambda_i,i=1,2$的特征向量,则由方程组$(A-\lambda_iI_2)v_i=0,i=1,2$可得
  \end{enuma}
    \[
      v_1 = \begin{pmatrix}
        1 \\
        \frac{d - a + \sqrt{(a - d)^2 + 4b^2} }{2b}
      \end{pmatrix}\quad \text{且}\quad
      v_2 = \begin{pmatrix}
        \frac{a - d - \sqrt{(a - d)^2 + 4b^2} }{2b} \\ 1
      \end{pmatrix}.
    \]
    我们将这两个向量除以它们的模长
    \[
      \| v_1 \| = \| v_2 \| = \SQRT{
        1 + \left( \frac{d - a + \sqrt{(a - d)^2 + 4b^2} }{2b} \right)^2
      },
    \]
    并取
    \[
      P = \left( \frac{v_1}{\|v_2\|}\bigg|
      \frac{v_2}{\|v_2\|} \right) = R_\theta,\;\text{其中}
      \,\tan\theta =
          \frac{d - a + \sqrt{(a-d)^2+4b^2}}{2b},b\ne0.
    \]
    定理得证.
\end{proof}

\section{Cayley--Hamilton定理的逆定理}

在这一节中,我们讨论Cayley--Hamilton定理的{\kaishu 逆定理}.
\begin{mybox}
  \begin{theorem}[Cayley--Hamilton定理的逆定理.]

    设$A\in\MM_2(\MC)$,且设$a,b\in\MC$满足$A^2-aA+bI_2=O_2$. 如果$A\notin\{\alpha I_2:\alpha\in\MC\}$,则$\Tr(A)=a$且$\det A=b$.
  \end{theorem}
\end{mybox}

\begin{proof}
  由Cayley--Hamilton定理我们有
  \begin{align*}
    & A^2 - aA + bI_2 = O_2, \\
    & A^2 - \Tr(A)A + (\det A)I_2 = O_2,
  \end{align*}
  由此得到$[a-\Tr(A)]A=(b-\det A)I_2$.

  如果$a-\Tr(A)\ne0$,我们得到
  \[
    A = \frac{b - \det A}{a - \Tr(A)}I_2,
  \]
  这与$A\notin\{\alpha I_2,\alpha\in\MC\}$矛盾.

  于是$a-\Tr(A)=0$,我们得到$b-\det A = 0$,定理得证.
\end{proof}

\begin{remark}
  值得一提的是,不存在矩阵$A\in\MM_2(\MC)$使得$A^2-aA+bI_2=O_2$对$a\ne\Tr(A)$和$b\ne\det A$成立. 要得到这个结论,我们令$A=\alpha I_2$,其中$\alpha\in\MC$满足方程$\alpha^2-a\alpha+b=0$. 则$\Tr(A)=2\alpha,\det A=\alpha^2$,如果$a,b$满足$a^2-4b\ne0$且$b\ne0$,我们有$a\ne\Tr(A)$且$b\ne\det A$.
\end{remark}

\section{矩阵$XY$与$YX$的特征多项式}

在这一节中,我们证明两个在矩阵论中关于矩阵$XY$和$YX$的基本结论.

\begin{theorem}[矩阵$XY$与$YX$的特征多项式.]

  如果$X,Y\in\MM_2(\MC)$,则矩阵$XY$与$YX$有相同的特征多项式,即$f_{XY}=f_{YX}$.
\end{theorem}

\begin{proof}
  由于$\Tr(XY)=\Tr(YX)$且$\det(XY)=\det(YX)$,我们有
  \[
    f_{XY}(x) = x^2 - \Tr(XY)x + \det(XY) = x^2 - \Tr(YX)x + \det(YX) = f_{YX}(x).
  \]
\end{proof}

\begin{nota}
  定理 \ref{thm2.7} 意味着下面的等式成立:
  \[
    \det (XY - \lambda I_2) = \det(YX - \lambda I_2),\,\forall X,Y\in\MM_2(\MC),\,\forall \lambda\in\MC.
  \]
  这个定理还说明矩阵$XY$与$YX$具有相同的特征值.
\end{nota}

下面的定理是定理 \ref{thm2.7} 的逆定理.

\begin{theorem}
  如果$A\in\MM_2(\MC)$满足
  \[
    \det (XY - A) = \det(YX - A),\,\forall X,Y\in\MM_2(\MC),
  \]
  则存在$a\in\MC$使得$A=aI_2$.
\end{theorem}

\begin{proof}
  设$E_{i,j}$表示$(i,j)$元等于1,而其他元均为0的矩阵,令$A=\begin{pmatrix}
    a & b \\
    c & d
  \end{pmatrix}\in\MM_2(\MC)$.

  如果$X=E_{1,2}$且$Y=E_{2,2}$,则$XY=E_{1,2},YX=O_2$,定理中假设的等式变为
  \[
    \det \begin{pmatrix}
      -a & 1 - b \\
      -c & -d
    \end{pmatrix} = \det
    \begin{pmatrix}
      -a & -b \\
      -c & -d
    \end{pmatrix},
  \]
  这意味着$c=0$.

  如果$X=E_{2,1},Y=E_{1,1}$,则$XY=E_{2,1}$且$YX=O_2$. 此时定理中假设的等式变为
  \[
    \det \begin{pmatrix}
      -a & -b \\
      1 - c & -d
    \end{pmatrix} = \det \begin{pmatrix}
      -a & -b \\
      -c & -d
    \end{pmatrix},
  \]
  这意味着$b=0$. 因此,$A=\begin{pmatrix}
    a & 0 \\
    0 & d
  \end{pmatrix}$.

  如果$X=E_{1,2},Y=E_{2,1}$,我们得到$XY=E_{1,1}$且$YX=E_{2,2}$,而条件
  \[
    \det \begin{pmatrix}
      1 - a & 0 \\
      0 & -d
    \end{pmatrix} = \det \begin{pmatrix}
      -a & 0 \\
      0 & 1 - d
    \end{pmatrix}
  \]
  则意味着$a=d$. 因此,$A=aI_2$,定理得证.
\end{proof}

现在我们给出前面定理的一个应用.

\begin{corollary}
  如果$A,B\in\MM_2(\MC)$是两个可逆矩阵,且满足
  \begin{equation}\label{eq2.5}
    \det (XAY + B) = \det(YBX + A),\,\forall X,Y\in\MM_2(\MC),
  \end{equation}
  则存在$a\in\MC^\ast$使得$A^2=B^2=aI_2$.
\end{corollary}

\begin{proof}
  如果$Y=O_2$,我们得到$\det A=\det B$. 如果$Y=I_2$,我们得到
  \[
    \det (XA + B) = \det (BX + A),\,\forall X\in\MM_2(\MC).
  \]
  由于$\det A=\det B$,这个等式左边左乘$\det B$,而右边右乘以$\det B$,我们有
  \begin{equation}\label{eq2.6}
    \det (BXA + B^2) = \det (BXA + A^2),\,\forall X\in\MM_2(\MC).
  \end{equation}

  由于矩阵$A$和$B$可逆,定义函数$f:\MM_2(\MC)\to\MM_2(\MC)$为$f(X)=BXA$,则它是满射,且等式 \eqref{eq2.6} 意味着
  \begin{equation}\label{eq2.7}
    \det (Z + B^2) = \det (Z + A^2),\,\forall Z\in\MM_2(\MC).
  \end{equation}
  在 \eqref{eq2.7} 中分别取$Z$等于$O_2,E_{1,1},E_{1,2},E_{2,1},E_{2,2}$,我们得到$B^2=A^2$. 现在在等式 \eqref{eq2.5} 左边右乘$\det B$,而在右边左乘$\det A$,我们有
  \[
    \det (XAYB + B^2) = \det (YBXA + A^2),\,\forall X,Y\in\MM_2(\MC),
  \]
  即
  \[
    \det (X_1Y_1 + C) = \det (Y_1X_1 + C),\,\forall X_1,Y_1\in\MM_2(\MC),
  \]
  其中$C=A^2=B^2$. 那么由定理 \ref{thm2.8},可知存在$a\in\MC^\ast$使得$C=aI_2$. 因此,$A^2=B^2=aI_2$,推论得证.
\end{proof}

\section{Jordan标准形}
\begin{mybox}
  \begin{theorem}[复Jordan标准形.]

    设$A\in\MM_2(\MC)$,且令$\lambda_1,\lambda_2$是$A$的特征值,则:
    \begin{enum}
      \item 如果$\lambda_1\ne\lambda_2$或者$A=\alpha I_2$对某个$\alpha\in\MC$成立,则存在可逆矩阵$P\in\MM_2(\MC)$使得
          \[
            A = P\begin{pmatrix}
              \lambda_1 & 0 \\
              0 & \lambda_2
            \end{pmatrix} P^{-1};
          \]
      \item 如果$\lambda_1=\lambda_2=\lambda$且$A\ne\lambda I_2$,则存在可逆矩阵$P\in\MM_2(\MC)$使得
          \[
            A = P\begin{pmatrix}
              \lambda & 1 \\
              0 & \lambda
            \end{pmatrix} P^{-1}.
          \]
    \end{enum}
  \end{theorem}
\end{mybox}

\begin{proof}
  \begin{enuma}
    \item 设$\lambda_1\ne\lambda_2$是矩阵$A=\begin{pmatrix}
          a & b \\
          c & d
        \end{pmatrix}$的特征值. 由于特征值互异,则$(a-d)^2+4bc\ne0$. 且存在$X_1=\begin{pmatrix}
          x_1 \\ y_1
        \end{pmatrix}\ne\begin{pmatrix}
          0 \\ 0
        \end{pmatrix}$,使得
  \end{enuma}
    \begin{equation}\label{eq2.8}
      AX_1 = \lambda_1X_2,
    \end{equation}
    且存在$X_2=\begin{pmatrix}
      x_2 \\ y_2
    \end{pmatrix}\ne\begin{pmatrix}
      0 \\ 0
    \end{pmatrix}$,使得
    \begin{equation}\label{eq2.9}
      AX_2 = \lambda_2X_2
    \end{equation}

    现在,我们注意到对任意$\alpha\in\MC$,有$X_2\ne\alpha X_1$,即相应于特征值$\lambda_1$和$\lambda_2$的特征向量是线性无关的. 否则,如果$X_2=\alpha X_1$对某个$\alpha\in\MC$成立,则意味着$AX_2=\alpha AX_1$,这就说明$\lambda_2X_2=\alpha\lambda_1X_1$, 于是$\alpha(\lambda_2-\lambda_1)X_1=0$. 由于$X_1\ne0$,因此$\lambda_1=\lambda_2$,矛盾. 因此$X_1,X_2$是线性无关的,这就意味着矩阵$P=(X_1|X_2)$是可逆的.

    等式 \eqref{eq2.8} 和 \eqref{eq2.9} 可以写为$A(X_1|X_2)=(\lambda_1X_1|\lambda_2X_2)$,即
    \[
      AP = P \begin{pmatrix}
        \lambda_1 & 0 \\
        0 & \lambda_2
      \end{pmatrix}\quad \Leftrightarrow\quad A = PJ_AP^{-1}\quad \text{其中}\quad
      J_A = \begin{pmatrix}
        \lambda_1 & 0 \\
        0 & \lambda_2
      \end{pmatrix} .
    \]
    \begin{enuma}
      \setcounter{enumi}{1}
      \item 现在我们考虑当$A$的特征值相等的情形,即$\lambda_1=\lambda_2=\lambda$且$A\ne\lambda I_2 $. 我们取向量$X_1=\begin{pmatrix}
            x_1 \\ y_1
          \end{pmatrix}\ne\begin{pmatrix}
            0 \\ 0
          \end{pmatrix}$,使得$AX_1=\lambda_1X_1$,以及取$X_1'=\begin{pmatrix}
            x_1' \\ y_1'
          \end{pmatrix}$,使得$AX_1'=\lambda X_1'+X_1$. 我们说明一下,这里$X_1$是相应于特征值$\lambda$的特征向量,而$X_1'$称为相应于特征值$\lambda$的{\kaishu 广义特征向量}\index{T!特征向量!广义特征向量}. 令$P=(X_1|X_1')$,则我们有
    \end{enuma}
          \[
            AP = A(X_1|X_1') = (\lambda X_1|\lambda X_1' + X_1),
          \]
    即
    \[
      AP = P\begin{pmatrix}
        \lambda & 1 \\
        0 & \lambda
      \end{pmatrix}\quad \Leftrightarrow\quad
      A = PJ_AP^{-1}\quad \text{其中}\quad
      J_A = \begin{pmatrix}
        \lambda & 1 \\
        0 & \lambda
      \end{pmatrix}.
    \]
    定理得证.
\end{proof}

\begin{mybox}
  \begin{remark}
    矩阵
    \[
      J_A = \begin{pmatrix}
        \lambda_1 & 0 \\
        0 & \lambda_2
      \end{pmatrix}\quad \text{或}\quad
      J_A = \begin{pmatrix}
        \lambda & 1 \\
        0 & \lambda
      \end{pmatrix}
    \]
    称为矩阵$A$的{\kaishu Jordan标准形}\index{B!标准形!Jordan标准形}. $P$的列向量$X_1,X_2$或者$X_1,X_1'$构成了$\MM_2(\MC)$的一组基,称为相应于矩阵$A$的\textbf{Jordan基}\index{J!Jordan基}. 由引理 \ref{lemma1.4},矩阵$P$是从标准基$\mathscr B=\{E_1,E_2\}$到Jordan基的过渡矩阵.
  \end{remark}
\end{mybox}

\begin{mybox}
  \begin{corollary}[特殊矩阵的Jordan基的形式.]
    \begin{itemize}
      \item 所有的幂零矩阵$A\in\MM_2(\MC),A\ne O_2$都形如
          \[
            A = P \begin{pmatrix}
              0 & 1 \\
              0 & 0
            \end{pmatrix}P^{-1},
          \]
          其中$P$是任意可逆矩阵.
      \item 所有的幂等矩阵$A\in\MM_2(\MC)$为$A=O_2,A=I_2$或者
          \[
            A = P\begin{pmatrix}
              1 & 0 \\
              0 & 0
            \end{pmatrix}P^{-1},
          \]
          其中$P$是任意可逆矩阵.
      \item 所有的对合矩阵$A\in\MM_2(\MC)$为$A=\pm I_2$或者
          \[
            A = P\begin{pmatrix}
              1 & 0 \\
              0 & -1
            \end{pmatrix}P^{-1},
          \]
          其中$P$是任意可逆矩阵.
      \item 所有的反对合矩阵$A\in\MM_2(\MC)$为$A=\pm\ii\cdot I_2$或者
          \[
            A = P\begin{pmatrix}
              0 & -1\\
              1 & 0
            \end{pmatrix}P^{-1},
          \]
          其中$P$是任意可逆矩阵.
    \end{itemize}
  \end{corollary}
\end{mybox}

现在我们讨论矩阵$A\in\MM_2(\MR)$的实标准形. 我们有下面的定理.
\begin{mybox}
  \begin{theorem}[实矩阵的实标准形.]
    \begin{enum}
      \item \label{thm2.10a}如果$A\in\MM_2(\MR)$且$\lambda_1,\lambda_2$是$A$的实特征值,则存在$P\in\MM_2(\MR)$使得
          \[
            A = P\begin{pmatrix}
              \lambda_1 & 0 \\
              0 & \lambda_2
            \end{pmatrix}P ^{-1}\quad\text{或者}\quad
            A = P\begin{pmatrix}
              \lambda & 1 \\
              0 & \lambda
            \end{pmatrix}P^{-1},
          \]
          上述情形要看$A$的特征值是否相同.
      \item 如果$A\in\MM_2(\MR)$,且$A$的特征值为$\lambda_1=\alpha+\ii\beta,\lambda_2=
          \alpha-\ii\beta,\alpha\in\MR,\beta\in\MR^\ast$,则存在可逆矩阵$P\in\MM_2(\MR)$使得
          \[
            A = P \begin{pmatrix}
              \alpha & \beta \\
              -\beta & \alpha
            \end{pmatrix} P^{-1}.
          \]
    \end{enum}
  \end{theorem}
\end{mybox}

\begin{proof}
  \begin{enuma}
    \item \ref{thm2.10a} 的证明类似于定理 \ref{thm2.9} 的证明.
    \item 我们指出,此时矩阵$A$的Jordan标准形为
  \end{enuma}
        \[
          J_A = \begin{pmatrix}
            \alpha + \ii\beta & 0 \\
            0 & \alpha - \ii\beta
          \end{pmatrix},
        \]
  且如果$AZ=\lambda_1Z,Z\ne0$,则$A\bar Z=\bar{\lambda_1}\bar Z=\lambda_2\bar Z$. 定义矩阵$P_{\MC}=(Z|\bar Z)$,则$A=P_{\MC}J_AP_{\MC}^{-1}$.

  如果$Z=X+\ii Y$,其中$X,Y$为实向量我们有
  \[
    AZ = \lambda_1Z \quad \Leftrightarrow\quad
    A(X + \ii Y) = (\alpha + \ii\beta )(X + \ii Y),
  \]
  于是我们得到等式$AX=\alpha X-\beta Y$和$AY=\beta X+\alpha Y$.

  定义矩阵$P=(X|Y)$,则我们有
  \[
    AP = A(X|Y) = (AX|AY) = (\alpha X-\beta Y|\beta X+\alpha Y) = P\begin{pmatrix}
      \alpha & \beta \\
      -\beta & \alpha
    \end{pmatrix},
  \]
  即$A=PJ_A^{\MR}P$,其中矩阵
  \[
    J_A^{\MR} = \begin{pmatrix}
      \alpha & \beta \\
      -\beta & \alpha
    \end{pmatrix}
  \]
  称为矩阵$A$的{\kaishu 实标准形}\index{B!标准形!实标准形}. 定理得证.
\end{proof}

现在我们给出矩阵$A\in\MM_2(\MQ)$的有理标准形. 我们有下面的定理.

\begin{mybox}
  \begin{theorem}[有理矩阵的有理标准形.]
    \begin{enum}
      \item\label{thm2.11a} 如果$A\in\MM_2(\MQ)$,且$\lambda_1,\lambda_2$是$A$的有理特征值,则存在矩阵$P\in\MM_2(\MQ)$,使得
          \[
            A = P\begin{pmatrix}
              \lambda_1 & 0 \\
              0 & \lambda_2
            \end{pmatrix}P^{-1} \quad\text{或者}\quad
            A = P\begin{pmatrix}
              \lambda & 1 \\
              0 & \lambda
            \end{pmatrix}P^{-1},
          \]
      上述情形取决于$A$的特征值是否相同.
      \item 如果$A\in\MM_2(\MQ)$,且$A$的特征值为$\lambda_1,\lambda_2\in\MC\backslash\MQ$,则$\lambda_1=\alpha+\sqrt\beta$且
          $\lambda_2=\alpha-\sqrt\beta,
          \alpha\in\MQ,\beta\in\MQ^\ast$,且存在可逆矩阵$P\in\MM_2(\MQ)$使得
          \[
            A = P \begin{pmatrix}
              \alpha & 1 \\
              \beta & \alpha
            \end{pmatrix}P^{-1}.
          \]
    \end{enum}
  \end{theorem}
\end{mybox}

\begin{proof}
  \begin{enuma}
    \item \ref{thm2.11a} 的证明类似于定理 \ref{thm2.9} 的证明.
    \item 此时矩阵$A$的Jordan标准形为
  \end{enuma}
  \[
      J_A = \begin{pmatrix}
        \alpha + \sqrt{\beta} & 0 \\
        0 & \alpha - \sqrt{\beta}
      \end{pmatrix}.
    \]

    如果$Z\ne0$是相应于特征值$\lambda_1=\alpha+\sqrt{\beta}$的特征向量,则$AZ=\lambda_1Z$. 令$Z=X+\sqrt\beta Y$,其中$X$和$Y$是有理向量. 直接计算得$A(X+\sqrt\beta Y)=(\alpha+\sqrt\beta)(X+\sqrt\beta Y)$,这意味着
    \[
      AX = \alpha X + \beta Y \quad \text{且}\quad
      AY = X + \alpha Y.
    \]
    这反过来又说明$A(X-\sqrt\beta Y)=(\alpha-\sqrt\beta)(X-\sqrt\beta Y)$,即$AZ'=\lambda_2Z'$,其中$Z'=X-\sqrt\beta Y$. 定义可逆矩阵$P_{\MC}=(Z|Z')$,则$A=P_{\MC}J_AP_{\MC}^{-1}$.

    令$P=(X|Y)\in\MM_2(\MQ)$,我们有
    \[
      AP = A(X|Y) = (\alpha X+\beta Y|X + \alpha Y) = (X|Y)\begin{pmatrix}
        \alpha & 1 \\
        \beta & \alpha
      \end{pmatrix} = P\begin{pmatrix}
        \alpha & 1 \\
        \beta & \alpha
      \end{pmatrix},
    \]
    即$A=PJ_A^{\MQ}P^{-1}$,其中矩阵
    \[
      J_A^{\MQ} = \begin{pmatrix}
        \alpha & 1 \\
        \beta & \alpha
      \end{pmatrix}
    \]
    称为矩阵$A$的{\kaishu 有理标准形}.\index{B!标准形!有理标准形} 定理得证.
\end{proof}

\section{问题}
\begin{problem}
  设$A\in\MM_2(\MC)$满足$\det A=1$,证明:$\det(A^2+A-I_2)+\det(A^2+I_2)=5$.
\end{problem}

\begin{problem}
  设$A\in\MM(Z)$满足$\det A=1$, 如果
  \[
    \det (A^2 - 3A + I_2) + \det (A^2 + A - I_2) = -4,
  \]
  求$\Tr(A)$.
\end{problem}

\begin{problem}
  设$A\in\MM_2(\MC)$满足$\Tr(A)=-1$,证明:
  \[
    \det (A^2 + 3A + 3I_2) - \det (A^2 + A) = 3.
  \]
\end{problem}

\begin{problem}
  设$a\in\MZ,a\ne\pm1$且$A\in\MM_2(\MZ)$. 证明:矩阵$aA+(a+1)I_2$和$aA-(a+1)I_2$是可逆的.
\end{problem}

\begin{problem}
  证明:任意矩阵$A\in\MM_2(\MZ)$可以写成两个可逆矩阵的和.
\end{problem}

\begin{problem}
  \footnote{这个问题指出任意奇异矩阵都是一列非奇异矩阵的极限.} 设$A\in\MM_2(\MC)$满足$\det A=0$,证明:存在一列矩阵$(A_n)_{n\in\MN}$,使得$\det A_n\ne0$且$\lim_{n\to\infty}A_n=A$.
\end{problem}

\begin{remark}[可逆矩阵的稠密性.]
  问题 \ref{problem2.6} 可以用来说明可逆矩阵的集合在所有矩阵中是稠密的.
\end{remark}

\begin{problem}
  设$A\in\MM_2(\MC)$,证明:$A$和$A_\ast$具有相同的特征多项式(相同的特征值).
\end{problem}

\begin{problem}
  如果一个矩阵的所有元都是非负实数,且每一行的和为1,则称之为{\kaishu 右随机矩阵}. \index{Y!右随机矩阵}证明:一个右随机矩阵$A\in\MM_2(\MC)$的最大特征值为1,最小特征值为$\Tr(A)-1$,并求出相应的特征向量.
\end{problem}

\begin{problem}
  证明:如果$A\in\MM_2(\MC)$的所有特征值为1,则对任意正整数$k$,$A$相似于$A^k$.
\end{problem}

\begin{problem}
  设$n\in\MN$,证明:如果$A,B\in\MM_2(\MC)$是相似矩阵,则$A^n$与$B^n$是相似矩阵. 这个命题的逆命题是否成立?
\end{problem}

\begin{problem}
  设$A=\begin{pmatrix}
    0 & 0 \\ 1 & 0
  \end{pmatrix}$.
  \begin{enum}
    \item 求出所有与$A$相似的矩阵$B\in\MM_2(\MR)$.
    \item 证明:$A$与$O_2$具有相同的特征多项式,但它们不相似.
  \end{enum}
\end{problem}

\begin{mybox}
\begin{problem}[两类特殊的相似矩阵.]
  \begin{enum}
    \item 证明:任意矩阵$A\in\MM_2(\MC)$与它的转置矩阵相似.
    \item\label{prob2.12b} 证明:任意矩阵$A\in\MM_2(\MC)$与一个对称矩阵相似.
  \end{enum}
  \begin{nota}
     问题的 \ref{prob2.12b} 可以约化为任意形如$\begin{pmatrix}
          \lambda & 1 \\
          0 & \lambda
        \end{pmatrix},\lambda\in\MC$的矩阵相似于一个复对称矩阵.
  \end{nota}
\end{problem}
\end{mybox}

\begin{mybox}
  \begin{problem}[转置矩阵与伴随矩阵矩阵相似.]

    证明:对任意$A\in\MM_2(\MC)$,存在$P\in\MM_2(\MC)$使得$A_\ast=PA\TT P^ {-1}$,并求出所有满足此性质的矩阵$P$.
  \end{problem}
\end{mybox}

\begin{mybox}
  \begin{problem}
    任意矩阵$A\in\MM_2(\MC)$都是两个对称矩阵的乘积.
  \end{problem}
\end{mybox}

\begin{problem}
  设$n\in\MN,n\ge2$,且设$A\in\MM_2(\MR)$. 证明:如果$A^n$是一个对称矩阵,且不形如$\alpha I_2,\alpha\in\MR$,则$A$是一个对称矩阵.
\end{problem}

\begin{problem}
  设$\MM$表示$\MM_2(\MC)$中特征值的模不超过1的矩阵的集合,证明:如果$A,B\in\MM$且$AB=BA$,则$AB\in\MM$.
\end{problem}

\begin{problem}
  设$A=\begin{pmatrix}
    2 & 5 \\
    -3 & 10
  \end{pmatrix}$且$B=\begin{pmatrix}
    3 & -2 \\
    4 & 9
  \end{pmatrix}$,证明:
  \[
    A^n - B^n = \frac{7^n-5^n}2(A-B),\quad \forall n\in\MN.
  \]
\end{problem}

\begin{problem}
  设$A,B\in\MM_2(\MR)$满足$AB=\begin{pmatrix}
    5 & 2 \\
    7 & 3
  \end{pmatrix}$,证明:$BA+A^{-1}B^{-1}=8I_2$.
\end{problem}

\begin{problem}
  设$A=\begin{pmatrix}
    1 & 3 \\
    3 & 10
  \end{pmatrix}$,且数列$(a_n)_{n\ge0}$递归定义为$a_{n+1}=3a_n+a_{n-1},n\ge1,a_0=0,a_1=1$.
  \begin{enum}
    \item 证明:$A^n=\begin{pmatrix}
      a_{2n-1} & a_{2n} \\
      a_{2n} & a_{2n+1}
    \end{pmatrix},n\ge1$.
    \item 如果数列$(x_n)_{n\ge0}$和$(y_n)_{n\ge0}$满足递推关系$\begin{pmatrix}
          x_{n+1} \\ y_{n+1}
        \end{pmatrix}=A\begin{pmatrix}
          x_n \\ y_n
        \end{pmatrix},n\ge0$,且$\begin{pmatrix}
          x_0 \\ y_0
        \end{pmatrix}=\begin{pmatrix}
          1 \\ 0
        \end{pmatrix}$,证明:$x_{n+1}^2+3x_{n+1}y_{n+1}-y_{n+1}^2=
        x_n^2+3x_ny_n-y_n^2$对任意$n\ge0$成立.
    \item 证明:如果自然数$x,y\in\MN$满足方程$x^2+3xy-y^2=1$,则存在$n\in\MN$使得$(x,y)=(a_{2n-1},a_{2n})$.
  \end{enum}
\end{problem}

\begin{problem}
  设$A,B\in\MM_2(\MR)$,且存在$n\in\MN$满足$(AB-BA)^n=I_2$. 证明:$(AB-BA)^4=I_2$,且$n$是偶数.
\end{problem}

\begin{problem}
  设整数$n\ge2$,且$A,B\in\MM_2(\MC)$满足$AB\ne BA$,而$(AB)^n=(BA)^n$. 证明:$(AB)^n=bI_2$对某个$b\in\MC$成立.
\end{problem}

\begin{problem}
  设$A\in\MM_2(\MR)$满足$\det(A^2-A+I_2)=0$.
  \begin{enum}
    \item\label{prob2.22a} 证明:$A^2-A+I_2=O_2$.
    \item 计算$\det(A^2+\alpha A+\beta I_2)$,其中$\alpha,\beta\in\MR$.
  \end{enum}
\end{problem}

\begin{problem}
  证明:任意矩阵$A\in\MM_2(\MR)$可以写为$A=B^2+C^2$,其中$B,C\in\MM_2(\MR)$. 如果我们附加条件$BC=CB$,这个结论是否成立?
\end{problem}

\begin{mybox}
  \begin{problem}
    设$A\in\MM_2(\MC)$,证明:
    \begin{enum}
      \item $\Re(\det A)=\det\Re(A)-\det\Im(A)$;
      \item $\Im(\det A)=\det\big(\Re(A)+\Im(A)\big)
          -\det\Re(A)-\det\Im(A)$.
    \end{enum}
  \end{problem}
\end{mybox}

\begin{mybox}
  \begin{problem}[一个极值问题.]

    设$\MM=\{A=(a_{ij})\in\MM_2(\MR):-1\le a_{ij}\le1,\forall i,j=1,2\}$,证明:$\max_{A,B\in\MM}\det(AB-BA)=16$.
  \end{problem}
\end{mybox}

\begin{problem}
  设$A,B\in\MM_2(\MC)$,证明:如果$\det(A+X)=\det(B+X)$对任意$X\in\MM_2(\MC)$成立,则$A=B$.
\end{problem}

\begin{problem}
  设$A,B\in\MM_2(\MR)$,证明:
  \[
    \det(A^2 + B^2 + AB - BA) = \det(A^2 + B^2) + \det(AB - BA).
  \]
\end{problem}

\begin{problem}
  设$A,B\in\MM_2(\MR)$,证明:$\det(A^2 + B^2)\ge \det(AB - BA)$.
\end{problem}

\begin{problem}
  设$A,B\in\MM_2(\MR)$,证明:如果$\det(AB+BA)\le0$,则$\det(A^2+B^2)\ge0$.
\end{problem}

\begin{problem}
  设$A,B\in\MM_2(\MR)$满足$A^2+B^2=O_2$且$AB=BA$,证明:$\det(A+B)=\det A+\det B$.
\end{problem}

\begin{problem}
  设$A,B\in\MM_2(\MR)$满足$A^2+B^2=O_2$且$AB=BA$,则
  \begin{enum}
    \item $\det A= \det B$;
    \item 如果$\det A\ne0$,则$A^2+B^2=O_2$.
  \end{enum}

  \begin{nota}
    如果$A,B\in\MM_2(\MR)$满足$A^2+B^2=O_2$且$AB=BA$,那么是得不到$A^2+B^2=O_2$的.
  \end{nota}
\end{problem}

\begin{problem}
  设$A,B\in\MM_2(\MR)$,证明:如果$AB=BA$且$\det(2A^2-3AB+2B^2)=0$,则$\det A=\det B$且$\det(A+B)=\frac72\det A$.
\end{problem}

\begin{problem}
  设$A,B\in\MM_2(\MQ)$是两个可交换的矩阵,满足$\det A=10$且$\det(A+\sqrt5B)=0$,求$\det(A^2-AB+B^2)$.
\end{problem}

\begin{problem}
  证明:对$\forall A,B\in\MM_2(\MC)$和$\forall a,b,c\in\MC$,有
  \[
    \det(aAB + bBA + cI_2) = \det (aBA + bAB + cI_2).
  \]
\end{problem}

\begin{problem}
  设$A\in\MM_2(\MR)$,且令
  \[
    f_A:\MM_2(\MR) \to \MM_2(\MR),\quad f_A(X) = \det(X + A) - \det(X - A).
  \]
  证明:
  \begin{enum}
    \item\label{prob2.35a} $f_{aA}=af_A,a\in\MR$;
    \item\label{prob2.35b} $f_{A+B}=f_A+f_B,B\in\MM_2(\MR)$;
    \item 存在数列$(x_n)_{n\ge1}$和$(y_n)_{n\ge1}$使得$f_{A^n}=x_nf_A+y_nf_{I_2}$.
  \end{enum}
\end{problem}

\begin{problem}
  证明:满足条件
  \begin{enum}
    \item $f(XY)=f(X)f(Y),\,\forall X,Y\in\MM_2(\MC)$
    \item $f(X+I_2)=f(X)+f(I_2)+\Tr(X),\,\forall X\in\MM_2(\MC)$
  \end{enum}
  的唯一函数$f:\MM_2(\MC)\to\MC$是行列式函数$f(X)=\det X$.
\end{problem}

\begin{problem}
  设$A\in\MM_2(\MR)$,且设函数$f:\MM_{2,1}(\MR)\to\MM_{2,1}(\MR)$定义为
  \[
    f_A\begin{pmatrix}
      x \\ y
    \end{pmatrix} = A\begin{pmatrix}
      x \\ y
    \end{pmatrix},\quad \begin{pmatrix}
      x \\ y
    \end{pmatrix} \in\MM_{2,1}(\MR).
  \]
  证明下面的结论是等价的:
  \begin{enum}
    \item\label{prob2.37a} $f_A$是单射;
    \item\label{prob2.37b} $f_A$是满射;
    \item\label{prob2.37c} $\det A\ne0$.
  \end{enum}
\end{problem}

\begin{problem}
  设$A\in\MM_2(\MZ)$,且令函数$f:\MM_{2,1}(\MZ)\to\MM_{2,1}(\MZ)$定义为
  \[
    f_A\begin{pmatrix}
      x \\ y
    \end{pmatrix} = A\begin{pmatrix}
      x \\ y
    \end{pmatrix},\quad \begin{pmatrix}
      x \\ y
    \end{pmatrix} \in\MM_{2,1}(\MZ).
  \]
  证明:
  \begin{enum}
    \item $f_A$是单射当且仅当$\det A\ne0$;
    \item $f_A$是满射当且仅当$\det A\in\{-1,1\}$.
  \end{enum}
\end{problem}

\begin{problem}[\kaishu 幂函数.]

  证明:对任意$n\in\MN,n\ge2$,函数$f:\MM_2(\MC)\to\MM_2(\MC),f(X)=X^n$既不是单射也不是满射.
\end{problem}

\begin{problem}[\kaishu 非满射的函数.]
  \begin{enum}
    \item 证明:函数$f:\MM_2(\MR)\to\MM_2(\MR),f(X)=X^{2016}+X^{2015}$不是满射.
    \item 证明:函数$f:\MM_2(\MR)\to\MM_2(\MR),f(X)=I_2+X+X^2
        +\cdots+X^{2016} $不是满射.
  \end{enum}
\end{problem}

\begin{problem}
   设$a,b,c,d\in(0,+\infty)$满足$ad-bc>0$,且令$A=\begin{pmatrix}
     a & - b\\
     -c & d
   \end{pmatrix}\in\MM_2(\MR)$. 如果一个矩阵$A\in\MM_2(\MR)$的的所有元是正实数,则称矩阵$A$是{\kaishu 正的},并且记为$A>0$. 证明:
   \begin{enum}
     \item 对任意正矩阵$X'$,存在一个正矩阵$X$,使得$AX=X'$;
     \item 存在$X>0$,使得$X'=AX>0$.
   \end{enum}
\end{problem}

\begin{problem}
   \cite{58} 设$A=\begin{pmatrix}
    a & b \\
    c & d
  \end{pmatrix}\in\MM_2(\MR)$,且$a>0,b>0,c>0,d>0$. 证明:$A$有一个特征向量$X=\begin{pmatrix}
    x \\ y
  \end{pmatrix}$满足$x>0$且$y>0$.
\end{problem}

\begin{problem}
  设$A,B\in\MM_2(\MR)$是所有元为正数的矩阵,证明:$(AB)^2=(BA)^2$当且仅当$AB=BA$.
\end{problem}

\begin{problem}
  设$A\in\MM_2(\MC)$满足$\Tr(A)=-1$且$\det A=1$,集合$\{A^n:n\in\MN\}$中有多少个元素?
\end{problem}

\begin{mybox}
  \begin{problem}[一个2016年的难题.]

   设整数$n\ge2$,且令$\mathscr P_n=\{X^n:X\in\MM_2(\MC)\}$. 证明:$\mathscr P_2=\mathscr P_n,\forall n\ge2$.

   这个问题将Seemous 2016, Protaras, Cyprus的问题2的(a)部分一般化了.
  \end{problem}
\end{mybox}

\begin{problem}
  设$A,B\in\MM_2(\MR)$满足$\det A=\det B=1$,证明:
  \begin{enum}
    \item $\Tr(AB)+\Tr(A^{-1}B)=\Tr(A)\Tr(B)$;
    \item $\Tr(BAB)+\Tr(A)=\Tr(B)\Tr(AB)$.
  \end{enum}
\end{problem}

\begin{problem}
  \begin{enumerate*}[label=(\alph*),listparindent=2em,itemjoin=\\]
    \item\label{prob2.47a} 设$A\in\MM_2(\MR)$满足$\Tr(A)>2$,证明:对任意$n\in\MN,A^n\ne I_2$.
    \item  设$r>0$,且$A\in\MM_2(\MR)$满足$\Tr(A)>2r$. 证明:对任意$n\in\MN,A^n\ne r^nI_2$.
  \end{enumerate*}
\end{problem}

\begin{problem}
  设$A\in\MM_2(\MR)$满足$\det A=1$且$|\Tr(A)|<2$,证明:对任意$n\ge2$,我们有$|\Tr(A^n)|\le2$,且存在$B_n\in\MM_2(\MR)$,使得$\det B_n=1,|\Tr(B_n)|<2$且$|\Tr(B_n^n)|\le2$.
\end{problem}

\begin{mybox}
  \begin{problem}
    设$A,B\in\MM_2(\MC)$满足$A\ne B$,且令$C=AB-BA$. 证明:$C$与$A$和$B$都可交换当且仅当$C=O_2$,即当且仅当$A$与$B$可交换.
  \end{problem}
\end{mybox}

\begin{problem}
  设$A\in\MM_2(\MC)$满足$\det(A-I_2)\in\MR$,且存在$n\in\MN$使得$A^n=I_2$. 证明:$\det(A-xI_2)\in\MR$对任意$x\in\MR$成立.
\end{problem}

\begin{problem}
  设$A,B\in\MM_2(\MC)$,且$n\ge2$是一个固定的整数. 证明:
  \begin{enum}
    \item 如果$(AB)^n=O_2$,则$(BA)^n=O_2$;
    \item 如果$(AB)^n=I_2$,则$(BA)^n=I_2$;
    \item 如果$AB\ne BA$,求矩阵$C\in\MM_2(\MC)$,使得只要$(AB)^n=C$就有$(BA)^n=C$.
  \end{enum}
\end{problem}

\begin{problem}
  设$A,B\in\GL_2(\MC)$且$\alpha,\beta\in\MC$满足$|\alpha|\ne|\beta|$,且$\alpha AB+\beta BA=I_2$,证明:$\det(AB-BA)=0$.
\end{problem}

\begin{problem}
  设矩阵$A,B,C\in\MM_2(\MR)$两两可交换,且$\det C=0$,证明:$\det(A^2+B^2+C^2)\ge0$.
\end{problem}

\begin{problem}
  设$A_0,A_1,\cdots,A_n\in\MM_2(\MR),n\ge2$是非零矩阵,满足$A_0\ne aI_2,\forall a\in\MR$且$A_0A_k=A_kA_0,\forall k=1,2,\cdots,n$. 证明:
  \begin{enum}
    \item $\det\left(\sum_{k=1}^nA_k^2\right)
        \ge0$;
    \item 如果$\det\left(\sum_{k=1}^nA_k^2\right) =0$,且$A_2\ne aA_1$对任意$a\in\MR$成立,则$\sum_{k=1}^nA_k^2=O_2$.
  \end{enum}
\end{problem}

\begin{problem}
  设$a\in(-1,1)$,且$A\in\MM_2(\MR)$满足$\det(A^4-aA^3-aA+I_2)=0$,证明:$\det A=1$.
\end{problem}

\begin{problem}
  设$B\in\MM_2(\MC)$是一个幂零矩阵,证明:如果$A\in\MM_2(\MC)$与$B$可交换,则$\det(A+B)=\det A$.
\end{problem}

\begin{problem}
  设$A,B\in\MM_2(\MC)$满足$AB=BA$,证明:如果存在整数$m,n\in\MN$使得$A^m=O_2$且$B^n=O_2$,则$AB=O_2$. 这个问题指出,{\kaishu 如果两个幂零矩阵可交换,则它们的乘积为零.}
\end{problem}

\begin{problem}[\kaishu 两个幂零矩阵的和(差)何时为幂零矩阵?]

  设$A,B\in\MM_2(\MC)$是两个非零的幂零矩阵,证明:$A\pm B$是一个幂零矩阵当且仅当$AB$与$BA$都是幂零矩阵.
\end{problem}

\begin{problem}
  设$A,B\in\MM_2(\MC)$,证明:$\Tr\big((AB)^2\big)
  =\Tr(A^2B^2)\Leftrightarrow (AB-BA)^2=O_2$.
\end{problem}

\begin{problem}[\kaishu 矩阵$AB-BA$何时幂零?]
  \begin{enum}
    \item \cite{28} 如果$A,B\in\MM_2(\MC)$满足$2015AB-2016BA=2017I_2$,则$(AB-BA)^2=O_2$。
    \item 一般地,设$m,n,p\in\MR,m\ne n$,且设$A,B\in\MM_2(\MC)$满足$mAB-nBA=pI_2$,证明:$(AB-BA)^2=O_2$.
  \end{enum}
\end{problem}

\begin{problem}
  设$A,B\in\MM_2(\MC)$,证明:$(AB)^2=AB^2A\Leftrightarrow (BA)^2=BA^2B$.
\end{problem}

\begin{problem}
  设$A,B\in\MM_2(\MC)$,证明:
  \[
    \det(A - B) \det (A + B) = \det (A^2 - B^2) \Leftrightarrow (AB - BA)^2 = O_2.
  \]
\end{problem}

\begin{problem}
  设$A,B\in\MM_2(\MR)$,证明:以下任意两个条件都可以推出第三个:
  \begin{enum}
    \item $\det(A^2 + B^2)=0$;
    \item $\det(AB - BA)=0$;
    \item $\det A = \det B = \frac12\det(A+B)$.
  \end{enum}
\end{problem}

\begin{problem}
  如果$A,B\in\MM_2(\MR)$满足$\det(AB+BA)=\det(AB-BA)$,则$\det(A^2+B^2)\ge0$.
\end{problem}

\begin{problem}
  如果$A,B\in\MM_2(\MC)$且$n\in\MN$,则$\det(A^n+B^n\pm AB)=\det(A^n+B^n\pm BA)$.
\end{problem}

\begin{problem}
  如果$A,B\in\MM_2(\MC)$且$A^2+B^2=AB$,则$(AB-BA)^2=O_2$.
\end{problem}

\begin{problem}
  如果$A,B\in\MM_2(\MC)$且$A^2=O_2$,则
  $\det(AB-BA)=0\Leftrightarrow \det(A+B)=\det B$.
\end{problem}

\begin{problem}
  设$A\in\MM_2(\MR)$满足$A^2=O_2$,证明:对$\forall B\in\MM_2(\MR)$,成立不等式$\det(AB-BA)\le0\le\det(AB+BA)$.
\end{problem}

\begin{problem}
  设$A,B\in\MM_2(\MC)\backslash\{O_2\}$满足$AB+BA=O_2$,证明:如果$\det(A-B)=0$,则$\Tr(A)=\Tr(B)=0$.
\end{problem}

\begin{problem}
  设$A,B\in\MM_2(\MC)$满足$\Tr(A)\Tr(B)=\Tr(AB)$,则
  \[
    \det(A^2 + B^2 + AB) = \det(A^2 + B^2) + \det (AB).
  \]
\end{problem}

\begin{mybox}
  \begin{problem}[幂零矩阵的中心化子.]

    设$A\in\MM_2(\MC)$,且令$\mathscr C(A)=\{X\in\MM_2(\MC):AX=XA\}$. 证明:
    \[
      A^2 = O_2 \quad \Leftrightarrow\quad
      |\det(A+X)| \ge |\det X|,\quad \forall X\in\mathscr C(A).
    \]
  \end{problem}
\end{mybox}

\begin{mybox}
  \begin{problem}
    \begin{enumerate*}[label=(\alph*),listparindent=2em,itemjoin=\\]
      \item 如果矩阵$A,B\in\MM_2(\MR)$满足$(A-B)^{-1}=A^{-1}-B^{-1}$,则$\det A=\det B=$ $\det(A-B)$.
      \item 如果$A,B\in\MM_2(\MC)$,结论还成立吗?
    \end{enumerate*}
  \end{problem}
\end{mybox}

\begin{mybox}
  \begin{problem}
    设$P$是一个实系数的无实根的多项式函数,且设$A\in\MM_2(\MR)$满足$\det P(A)=0$,证明:$P(A)=O_2$.
  \end{problem}
\end{mybox}

\begin{problem}
  \cite[p.145]{58} 设矩阵$A,B\in\MM_2(\MR)$满足$A^2=B^2=I_2$且$AB+BA=O_2$,证明:存在可逆矩阵$Q\in\MM_2(\MR)$使得
  \[
    Q^{-1}AQ = \begin{pmatrix}
      1 & 0 \\
      0 & -1
    \end{pmatrix}\quad \text{且}\quad
    Q^{-1}BQ = \begin{pmatrix}
      0 & 1 \\
      1 & 0
    \end{pmatrix}.
  \]
\end{problem}

\begin{problem}
  设$A,B\in\MM_2(\MC)$满足$AB=O_2$,证明:
  \[
    \det (A + B)^n = \det (A^n + B^n) ,\quad \forall n\ge1.
  \]
\end{problem}

\begin{problem}
  \begin{inparaenum}[(a)]
    \item 证明:存在矩阵$A,B\in\MM_2(\MR)$,使得
      \[
        \det (xA + yB) = x^2 + y^2,\quad \forall x,y\in\MR.
      \]
    \item 证明:不存在矩阵$A,B,C\in\MM_2(\MR)$,使得
      \[
        \det (xA + yB + zC) = x^2 + y^2 + z^2,\quad\forall x,y,z\in\MR.
      \]
  \end{inparaenum}
\end{problem}

\section{解答}
\begin{solution}
  令$f_A(x)=\det(A-xI_2)=x^2-tx+1$,其中$t=\Tr(A)$. 由定理 \ref{thm2.2},我们有$A^2=tA-I_2$,这意味着$A^2+A-I_2=(t+1)A-2I_2$且$A^2+I_2=tA$. 直接计算得
  \begin{align*}
    \det(A^2 + A - I_2) + \det (A^2 + I_2) & =
    (t+1)^2f_A\Big(\frac2{t + 2}\Big) + t^2 \\
    & = (t+1)^2 \bigg[ \Big( \frac2{t+1} \Big)^2
    - \frac{2t}{t+1} + 1 \bigg] + t^2 \\
    & = 5.
  \end{align*}
\end{solution}

\begin{solution}
  $\Tr(A)=3$,见问题 \ref{problem2.1} 的解答.
\end{solution}

\begin{solution}
  令$f_A(x)=\det(A-xI_2)=x^2+x+d$,其中$d=\det A$. 定理 \ref{thm2.2} 说明$A^2=-A-dI_2$,且这意味着$A^2+3A+3I_2=2A-(d-3)I_2$以及$A^2+A=-dI_2$. 于是
  \begin{align*}
    \det (A^2 + 3A + 3I_2) - \det(A^2 + A) & =
    4f_A\Big( \frac{d-3}2 \Big) - d^2 \\
    & = 4\bigg[ \Big( \frac{d-3}2 \Big)^2 +
    \frac{d-3}2 + d \bigg] - d^2 \\
    & = 3.
  \end{align*}
\end{solution}

\begin{solution}
  由推论 \ref{coro2.4},我们有$\det\big(aA+(a+1)I_2\big)=a^2\det A+a(a+1)\alpha+(a+1)^2$对某个$\alpha\in\MZ$成立. 如果$\det\big(aA+(a+1)I_2\big)=0$,则$a^2\det A+a(a+1)\alpha+(a+1)^2=0\Rightarrow a|(a+1)^2$. 于是有$a^2+2a+1=ab$对某个$b\in\MZ$成立, 然而,这个等式意味着$a|1$,这与$a\ne \pm1$矛盾.
\end{solution}

\begin{solution}
  设$\lambda\notin\Spec(A),\lambda\ne0$,且令$B=A-\lambda I_2,C=\lambda I_2$.
\end{solution}

\begin{solution}
  由于$\det A=0$,我们得到$0\in\Spec(A)$. 设$\lambda_n$是一列实数或者复数,满足$\lim_{n\to\infty}\lambda_n=0$且$\lambda_n\notin\Spec(A)$. 令$A_n=A-\lambda_nI_2$,我们有$\lim_{n\to\infty}A_n=A$,且由于$\lambda_n\notin\Spec(A)$,则 $\det(A_n)=\det(A-\lambda_nI_2)\ne0$.
\end{solution}

\begin{solution}
  我们有
  \[
    f_{A_\ast}(x) = \det (A - xI_2) = \begin{vmatrix}
      d - x & -b \\
      -c & a - x
    \end{vmatrix} = x^2 - (a + d)x + ad - bc = f_A(x).
  \]
\end{solution}

\setcounter{solution}{8}

\begin{solution}
  设$J_A$是$A$的Jordan标准形,且$P$是一个可逆矩阵,满足$A=PJ_AP^{-1}$. 如果$J_A=I_2$,结论显然. 如果$J_A=\begin{pmatrix}
    1 & 1 \\
    0 & 1
  \end{pmatrix}$,则$A^k=PJ_A^kP^{-1}$. 直接计算可知$\begin{pmatrix}
    1 & k \\
    0 & 1
  \end{pmatrix}=J_A^k=Q^{-1}J_AQ$,其中$Q=\begin{pmatrix}
    1 & 0 \\
    0 & k
  \end{pmatrix}$. 这意味着
  \[
    A^k = PQ^{-1}J_AQP^{-1} = (PQP^{-1})^{-1}PJ_AP^{-1}(PQP^{-1}) =
    (PQP^{-1})^{-1}A(PQP^{-1}),
  \]
  这说明$A^k\sim A$.
\end{solution}

\begin{solution}
  $A\sim B\Rightarrow \exists P\in\GL_2(\MC)$,使得$B=P^{-1}AP$,这意味着$B^n=P^{-1}A^nP$,这就证明了$A^n\sim B^n$.

  其逆命题是不成立的. 设$n=2,A=\begin{pmatrix}
    0 & 0 \\
    1 & 0
  \end{pmatrix}$且$B=O_2$,则$A^2=B^2=O_2$,于是$A^2\sim B^2$,但$A$和$B$是不相似的.
\end{solution}

\setcounter{solution}{11}

\begin{solution}
  \begin{inparaenum}[(a)]
    \item 只需要对Jordan标准形证明即可. 如果$J_A=\begin{pmatrix}
          \lambda_1 & 0 \\
          0 & \lambda_2
        \end{pmatrix}$,结论显然. 如果$J_A=\begin{pmatrix}
          \lambda & 1 \\
          0 & \lambda
        \end{pmatrix}$,我们令$P=\begin{pmatrix}
          0 & 1 \\
          1 & 0
        \end{pmatrix}$,则我们有$P^{-1}J_AP=J_A^T$.

    \item 只需要对Jordan标准形证明即可. 如果$J_A=\begin{pmatrix}
          \lambda_1 & 0 \\
          0 & \lambda_2
        \end{pmatrix}$,结论显然. 如果$J_A=\begin{pmatrix}
          \lambda & 1 \\
          0 & \lambda
        \end{pmatrix}$,我们需要找到一个可逆矩阵$Q\in\MM_2(\MC)$和一个对称矩阵$B\in\MM_2(\MC)$,使得$Q^{-1}J_AQ=B$. 令$Q=\begin{pmatrix}
          a & b \\
          c & d
        \end{pmatrix}$,且设$\varDelta=ad-bc\ne0$. 直接计算可知
        \[
          B = \frac1{\varDelta} \begin{pmatrix}
            cd + \lambda\varDelta & d^2 \\
            -c^2 & -cd + \lambda\varDelta
          \end{pmatrix},
        \]
        且由于$B$是对称矩阵,我们得到$c^2+d^2=0$. 注意到$B$不可能是实矩阵,令$d=\ii,c=1,a=\ii.b=0$,我们有
        \[
          Q = \begin{pmatrix}
            \ii & 0 \\
            1 & \ii
          \end{pmatrix}\quad \text{且}\quad
          B = \begin{pmatrix}
            \lambda - \ii & 1 \\
            1 & \lambda + \ii
          \end{pmatrix}.
        \]
  \end{inparaenum}
\end{solution}

\begin{solution}
  令$A=\begin{pmatrix}
    a & b \\
    c & d
  \end{pmatrix}$,则$A_\ast=\begin{pmatrix}
    d & -b \\
    -c & a
  \end{pmatrix}$. 如果$P=\begin{pmatrix}
    0 & -1 \\
    1 & 0
  \end{pmatrix}$,则$PA\TT P^{-1}=A_\ast$.

  如果$Q$是另一个矩阵,使得$QA\TT Q^{-1}=A_\ast,\,\forall A\in\MM_2(\MC)$,则
  \[
    PA\TT P^{-1}=QA\TT Q^{-1}\Rightarrow (Q^{-1}P)A\TT=A\TT(Q^{-1}P),\,\forall A\in\MM_2(\MC),
  \]
  这意味着$Q^{-1}P$与所有$\MM_2(\MC)$中的矩阵可交换. 由定理 \ref{thm1.1},$Q^{-1}P=\alpha I_2$对某个$\alpha\in\MC$成立,这意味着$Q=\begin{pmatrix}
    0 & -\beta \\
    \beta & 0
  \end{pmatrix}$,其中$\beta\in\MC^\ast$.
\end{solution}

\begin{solution}
  设$J_A$是$A$的Jordan标准形,可逆矩阵$P$满足$A=PJ_AP^{-1}$. 如果$J_A=\begin{pmatrix}
    \lambda_A & 0 \\
    0 & 1
  \end{pmatrix}$,则$J_A=BC$,其中
  \[
    B = \begin{pmatrix}
      \lambda_1 & 0 \\
      0 & 1
    \end{pmatrix} \quad \text{且} \quad
    C = \begin{pmatrix}
      1 & 0 \\
      0 & \lambda_2
    \end{pmatrix}.
  \]
  我们有$A=PJ_AP^{-1}=[PBP\TT][(P^{-1})\TT CP^{-1}]$,其中$PBP\TT$和$(P^{-1})\TT CP^{-1}$都是对称矩阵.

  如果$J_A=\begin{pmatrix}
    \lambda & 1 \\
    0 & \lambda
  \end{pmatrix}$,则$J_A = BC$,其中
  \[
    B = \begin{pmatrix}
      1 & 1 \\
      0 & 1
    \end{pmatrix} \quad \text{且} \quad
    C = \begin{pmatrix}
      0 & \lambda \\
      \lambda & 1 - \lambda
    \end{pmatrix}.
  \]
  我们有$A=PJ_AP^{-1}=[PBP\TT][(P^{-1})\TT CP^{-1}]$,其中$PBP\TT$和$(P^{-1})\TT CP^{-1}$都是对称矩阵.
\end{solution}

\setcounter{solution}{15}

\begin{solution}
  设$A,B\in\MM$. 如果$\lambda_A$是$A$的一个特征值,$\lambda_B$是$B$的一个特征值. 由于$AB=BA$,由定理 \ref{thm2.1},我们得到$\lambda_{AB}=\lambda_A\lambda_B$,于是$|\lambda_A\lambda_B|=|\lambda_A||\lambda_B|\le1$.
\end{solution}

\begin{solution}
  $A$和$B$的特征值都是7和5. 由定理 \ref{thm3.1},我们有$A^n=7^nD+5^nC$,其中$D=\frac{A-5I_2}2,C=\frac{7I_2-A}2$;而$B^n=7^nU+5^nV$,其中$U=\frac{B-5I_2}2,V=\frac{7I_2-B}2$. 由此即可得到$A^n-B^n=\frac12(7^n-5^n)(A-B)$.
\end{solution}

\begin{solution}
  矩阵$BA$的特征多项式为$f_{BA}(x)=f_{AB}(x)=x^2-8x+1$. 由定理 \ref{thm2.2},我们有$(BA)^2-8BA+I_2=O_2$. 将这个等式两边乘以$(BA)^{-1}=A^{-1}B^{-1}$,我们得到$BA+A^{-1}B^{-1}=8I_2$.
\end{solution}

\setcounter{solution}{19}

\begin{solution}
  我们有$\Tr(AB-BA)=0$且$\det\nolimits^n(AB-BA)=1\Rightarrow
  \det(AB-BA)=\pm1$. 我们对矩阵$AB-BA$应用Cayley--Hamilton定理可得$(AB-BA)^2=\pm I_2$,这意味着$(AB-BA)^4=I_2$.

  利用反证法,我们假定$n=2k+1$. 由于$(AB-BA)^n=I_2$,我们有$\det(AB-BA)=-1$,那么根据Cayley--Hamilton定理可知$(AB-BA)^2=-I_2$. 因此,$I_2=(AB-BA)^{2k+1}=(-I_2)^k(AB-BA)$,这意味着$AB-BA=(-1)^kI_2$,这就与$\Tr(AB-BA)=0$矛盾了.
\end{solution}


\begin{solution}
  设$f\in\MC[x]$是矩阵$AB$和$BA$的特征多项式. 用$f$除$x^n$,我们可得存在$Q\in\MC[x]$以及$a,b\in\MC$,使得$x^n=f(x)Q(x)+ax+b$. 将$x$替换为$AB$和$BA$,我们得到$(AB)^n=aAB+bI_2$且$(BA)^n=aBA+bI_2$. 由于$(AB)^n=(BA)^n$且$AB\ne BA$,我们得到$a=0$,而这反过来又说明$(AB)^n=(BA)^n=bI_2$.
\end{solution}

\begin{solution}
  \begin{inparaenum}[(a)]
    \item 注意到$A^2-A+I_2=(A-\varepsilon I_2)(A-\bar\varepsilon I_2)$,其中$\varepsilon^2-\varepsilon+1=0,
        \varepsilon\in\MC\backslash\MR$. 由于$\det(A^2-A+I_2)=0$,我们有$\det(A-\varepsilon I_2)=0$或$\det(A-\bar\varepsilon I_2)=0$. 令$f(x)=\det(A-xI_2)\in\MR[x]$是$A$的特征多项式. 如果$\varepsilon$或$\bar\varepsilon$是$f$的一个根,由于$f$是实系数的,那么$\varepsilon$与$\bar\varepsilon$都是$f$的根. 这意味着$f(x)=(x-\varepsilon)(x-\bar\varepsilon)=x^2-x+1$,那么由Cayley--Hamilton定理可知$A^2-A+I_2=O_2$.

    \item 我们有$\det(A^2+\alpha A+\beta I_2)=\det[(\alpha+1)A+(\beta-1)I_2]$,下面分$\alpha=-1$和$\alpha\ne-1$两种情形.

        如果$\alpha=-1$,由 \ref{prob2.22a} 可知$A^2-A+\beta I_2=(\beta-1)I_2$,这就意味着$\det(A^2-A+\beta I_2)=(\beta-1)^2$.

        如果$\alpha=-1$,我们有
        \begin{align*}
          \det(A^2 + \alpha A + \beta I_2) & = (\alpha + 1)^2 \det \Big( A - \frac{1-\beta}{\alpha+1} I_2\Big) \\
          & = (\alpha + 1)^2 f\Big( \frac{1-\beta}{\alpha+1} \Big) \\
          & = (\alpha + 1)^2 \bigg[
            \Big( \frac{1-\beta}{\alpha+1} \Big)^2 - \frac{1-\beta}{\alpha+1} + 1
          \bigg] \\
          & = \alpha^2 + \beta^2 + \alpha\beta + \alpha - \beta + 1.
        \end{align*}
  \end{inparaenum}
\end{solution}

\begin{solution}
  \begin{inparaenum}[(a)]
    \item 设$A=\alpha I_2$,其中$\alpha\in \MR$. 如果$\alpha<0$,则$B=C=\sqrt{-\frac\alpha2}\begin{pmatrix}
          0 & 1 \\
          -1 & 0
        \end{pmatrix}$. 如果$\alpha=0$,则$B=C=O_2$. 如果$\alpha>0$,则$B=C=\sqrt{\frac\alpha2}I_2$.

        现在我们考虑$A\ne\alpha I_2,\forall \alpha\in\MR$的情形. 令$t=\Tr(A),d=\det(A)$,则我们有$A^2-tA+dl_2=O_2$. 我们要找出$\alpha,\beta,\gamma\in\MR$,使得$A=(\alpha A+\beta I_2)^2+\gamma I_2$. 计算可得$A=(\alpha^2t+2\alpha\beta)A+(\beta^2+\gamma-\alpha^2
        d)I_2$. 这意味着$\alpha^2t+2\alpha\beta=1$且
        $\beta^2+\gamma-\alpha^2d=0$. 我们取$\alpha=1,\beta=\frac{1-t}2,\gamma=d-\frac{(1-t)^2}4$.

  \end{inparaenum}
        \begin{itemize}
          \item 如果$\gamma>0$,则$B=\alpha A+\beta I_2$且$C=\sqrt\gamma I_2$.
          \item 如果$\gamma=0$,则$B=C=\frac1{\sqrt2}(\alpha
              A+\beta I_2)$.
          \item 如果$\gamma<0$,则$B=\alpha A+\beta I_2$且$C=\sqrt{-\gamma}\begin{pmatrix}
                0 & 1 \\
                -1 & 0
              \end{pmatrix}$.
        \end{itemize}

  \begin{enuma}
    \setcounter{enumi}{1}
    \item 如果$BC=CB$,则$B^2+C^2=(B+\ii C)(B-\ii C),\det (B^2+C^2)=|\det(B+\ii C)|^2\ge0$,所以不存在矩阵$A,\det A<0$,使得$A=B^2+C^2$且$BC=CB$.
  \end{enuma}
\end{solution}

\begin{solution}
  注意到$A=\Re(A)+\ii\Im(A)$,然后利用推论 \ref{coro2.4},取$x=1$和$y=\ii$即可.
\end{solution}

\begin{solution}
  利用引理 \ref{lemma2.10},我们有$\det(AB-BA)=\Tr(A^2B^2)-\Tr\big((AB)^2\big)$. 由定理 \ref{thm2.2},计算可得$\Tr(A^2B^2)-\Tr
  \big((AB)^2\big)=-t_{AB}^2+t_At_Bt_{AB}-t_A^2d_B
  -d_At_B^2+4d_Ad_B$,这是一个$t_{AB}$的二次函数. 此函数的判别式$\varDelta=(t_A^2-4d_A)(t_B^2-4d_B)$,且它的最大值为$\frac{\varDelta}4=\frac14(t_A^2-4d_A)(t_B^2-4d_B)$. 如果$A=\begin{pmatrix}
    a & b \\
    c & d
  \end{pmatrix}\in\MM_2(\MR)$,
  则$t_A^2-4d_A=(a-d)^2+4bc\le8$,这是因为$a,b,c,d\in[-1,1]$. 于是$\max_{A,B\in\MM}\det(AB-BA)=16$,等号在$A=\begin{pmatrix}
    1 & -1 \\
    -1 & 1
  \end{pmatrix}$和$B=\begin{pmatrix}
    -1 & -1 \\
    -1 & 1
  \end{pmatrix}$时取到.
\end{solution}

\begin{solution}
  设$E_{ij},i,j=1,2$表示$(i,j)$元为1,而其他元为0的矩阵,令$X=O_2,E_{11},E_{12},E_{21}$和$E_{22}$.
\end{solution}

\begin{solution}\hspace*{-0.5em}
  和 \textbf{\ref{problem2.28}}\hspace*{0.5em}
    我们有
    \[
      |\det(A+\ii B)|^2 = \det(A + \ii B) \det(A - \ii B) = \det[A^2 + B^2 - \ii(AB - BA)].
    \]
    于是由推论 \ref{coro2.4} 取$x=1,y=-\ii$可得
    \begin{align*}
      \det &[A^2 + B^2 - \ii(AB - BA)] = \det(A^2 + B^2) - \det(AB - BA) \\
      &-\ii [\det(A^2 + B^2 + AB - BA) - \det(A^2 + B^2) - \det(AB - BA)].
    \end{align*}
    由于$\det[A^2 + B^2 - \ii(AB - BA)]=|\det(A+\ii B)|^2\ge0$,我们得到$\det(A^2+B^2)\ge\det(AB-BA)$,且$\det(A^2 + B^2 + AB - BA) - \det(A^2 + B^2) - \det(AB - BA)=0$.
\end{solution}

\setcounter{solution}{28}

\begin{solution}
  令$f(x)=\det[A^2+B^2+x(AB+BA)]\in\MR[x]$,我们有
  \begin{gather*}
    f(1) = \det (A + B)^2 = \det{}^2(A + B) \ge0,\\
    f(-1) = \det (A - B)^2 = \det{}^2(A - B) \ge 0,\\
    f(x) = \det (A^2 + B^2) + \alpha x + x^2 \det(AB + BA),\alpha\in\MR.
  \end{gather*}

  如果$\det(AB+BA)=0$,则$f$是一个线性的单调函数(或常函数),且由于$0$介于$-1$与$1$之间,我们得到$f(0)\ge0$.
\end{solution}

\begin{solution}
  我们有$0=\det(A^2+B^2)=\det(A+\ii B)\det(A-\ii B)=\det(A+\ii B)\det(A-\ii B)$,于是$\det(A+\ii B)=0$或$\det(A-\ii B)=0$. 由推论 \ref{coro2.4} 我们得到
  \[
    \det (A \pm \ii B) = \det A - \det B \pm \ii
    [ \det(A + B) - \det A - \det B ].
  \]
  由于$\det(A\pm B)=0$,我们有$\det A=\det B$且$\det(A+B)=\det A+\det B$.
\end{solution}

\begin{solution}
  \begin{inparaenum}[(a)]
    \item 这一部分见问题 \ref{problem2.30} 的解答.

    \item 如果$\det A\ne0$,则$A^2+B^2=A^2(I_2+C^2)$,其中$C=A^{-1}B\in\MM_2(\MR)$. 我们有$\det(A^2+B^2)=0\Leftrightarrow\det(I_2+C^2)=0
        \Leftrightarrow\det(C+\ii I_2)(C-\ii I_2)=0$. 利用在问题 \ref{problem2.30} 中的技巧,我们得到$\det C=1$且$\det(C+I_2)=\det C+1$. 由于$\det(C+I_2)=\det C+\Tr(C)+1$,那么可得$\Tr(C)=0$. 对矩阵$C$应用Cayley-Hamilton定理可得$C^2+I_2=O_2$,这意味着$A^2+B^2=A^2(I_2+C^2)=O_2$.
  \end{inparaenum}
\end{solution}

\begin{solution}
  设$\alpha=\frac{3+\ii\sqrt7}4$,且注意到$2A^2-3AB+2B^2=2(A-\alpha B)(A-\bar \alpha B)$. 我们有$0=\det(2A^2-3AB+2B^2)=4|\det(A-\alpha B)|^2$,这意味着$\det(A-\alpha B)=0$. 由推论 \ref{coro2.4},我们有
  \[
    \det(A - \alpha B) = \det A + \alpha^2\det B - \alpha [\det(A + B) - \det A - \det B].
  \]
  进一步,由$\alpha^2=\frac32\alpha-1$,可得
  \[
    \det (A - \alpha B) = \det A - \det B + \alpha
    \Big[ \det A + \frac52\det B - \det(A + B) \Big] = 0.
  \]

  由于$\alpha\notin\MR$,我们有$\det A + \frac52\det B-\det(A+B)=0$,且$\det A=\det B$,这就意味着$\det(A+B)=\frac72\det A$.
\end{solution}
\begin{remark}
  如果$A,B\in\MM_2(\MR)$是可交换的矩阵,满足$\det A\ne0$或$\det B\ne0$,且$\det(2A^2-3AB+2B^2)=0$,则$2A^2-3AB+2B^2=O_2$
  (证明见问题 \ref{problem2.31} 的解答).
\end{remark}

\begin{solution}
  $\det(A^2-AB+B^2)=124$.
\end{solution}

\begin{solution}
  令$f(x,y)=\det(xAB+yBA+cI_2),g(x,y)=\det(xBA+yAB+cI_2)$. 我们注意到$f$和$g$都是关于变量$x$与$y$的不超过二次的多项式,且具有形式
  \begin{gather*}
    f(x,y) = a_{11}x^2 + a_{12}xy + a_{22}y^2 + a_1x + a_2y + a_3 , \\
    g(x,y) = b_{11}x^2 + b_{12}xy + b_{22}y^2 + b_1x + b_2y + b_3 .
  \end{gather*}
  由于$f(x,y)=g(y,x)$,我们得到$a_{11}=b_{22},a_{12}=b_{12},a_{22}=b_{11},a_1=b_2,a_2=b_1$,
  且$a_3=b_3$. 于是有
  \begin{gather*}
    f(x,y) = a_{11}x^2 + a_{12}xy + a_{22}y^2 + a_1x + a_2y + a_3 , \\
    g(x,y) = a_{22}x^2 + a_{12}xy + a_{11}y^2 + a_2x + a_1y + a_3 .
  \end{gather*}
  我们有
  \begin{gather*}
    f(x,0) = \det (xAB + cI_2) = x^2\det(AB) + cx\Tr(AB) + c^2, \\
    g(x,0) = \det (xBA + cI_2) = x^2\det(BA) + cx\Tr(BA) + c^2,
  \end{gather*}
  所以$f(x,0)=g(x,0),\forall x\in\MC$,于是$a_{11}=a_{22}$且$a_1=a_2$. 因此,
  \[
    f(x,y) = g(x,y) = a_{11}(x^2+y^2) + a_{12}xy + a_1(x+y) + a_3,\quad \forall x,y\in\MC.
  \]
\end{solution}

\begin{solution}
  设$A_1,A_2$是$A$的列向量,$X_1,X_2$是$X$的列向量,即$A=(A_1|A_2)$且$X=(X_1|X_2)$. 计算可得
  \begin{align*}
    f_A(X) = {}& \det (A_1 + X_1|A_2 + X_2) - \det(X_1-A_1|X_2 - A_2) \\
    = {}& \det (A_1|A_2) + \det(A_1|X_2) + \det(X_1|A_2) + \det (X_1|X_2) \\
      & - \det(X_1|X_2) + \det(X_1|A_2) + \det(A_1|X_2) - \det(A_1|A_2) \\
    = {}& 2[\det(A_1|X_2) + \det(X_1|A_2)].
  \end{align*}

  \begin{inparaenum}[(a)]
    \item 我们有
      \begin{align*}
        f_{aA}(X) & = 2 [ \det(aA_1|X_2) + \det(X_1|aA_2) ] \\
        & = 2a[ \det(A_1|X_2) + \det(X_1|A_2) ] \\
        & = af_A(X).
      \end{align*}

    \item 我们有
      \begin{align*}
        f_{A+B}(X) & = 2[ \det(A_1+B_1|X_2) + \det(X_1|A_2+B_2) ] \\
        & =  2 [ \det(A_1|X_2) + \det(X_1|A_2) ] +
             2 [ \det(B_1|X_2) + \det(X_1|B_2) ] \\
        & = f_A(X) + f_B(X).
      \end{align*}

    \item 由定理 \ref{thm3.2},存在数列$(x_n)_{n\ge1}$和$(y_n)_{n\ge1}$,使得$A^n=x_nA+y_nI_2,\forall n\ge1$. 由 \ref{prob2.35a} 和 \ref{prob2.35b} 可得
        \[
          f_{A^n} = f_{x_nA+y_nI_2} = f_{x_nA} + f_{y_nI_2} = x_nf_A + y_nf_{I_2}.
        \]
  \end{inparaenum}

  我们顺带一提$f_{I_2}(X)=2\Tr(X)$.
\end{solution}

\setcounter{solution}{36}

\begin{solution}
  我们证明 \ref{prob2.37a} $\Leftrightarrow$ \ref{prob2.37c} 以及 \ref{prob2.37b} $\Leftrightarrow$ \ref{prob2.37c}. 我们有
  \[
    f_A\begin{pmatrix}
      x_1 \\ y_1
    \end{pmatrix} = f_A\begin{pmatrix}
      x_2 \\ y_2
    \end{pmatrix}\quad \Leftrightarrow \quad
    A \begin{pmatrix}
      x_1 - x_2 \\
      y_1 - y_2
    \end{pmatrix} = \begin{pmatrix}
      0 \\ 0
    \end{pmatrix},
  \]
  这是一个含有两个变量两个方程的齐次线性方程组. 此方程只有零解$x_1-x_2=0,y_1-y_2=0$当且仅当$\det A\ne0$. 因此,
  \[
    \det A \ne 0 \quad \Leftrightarrow \quad
    \begin{pmatrix}
      x_1 \\ y_1
    \end{pmatrix} = \begin{pmatrix}
      x_2 \\ y_2
    \end{pmatrix}.
  \]

  现在我们考虑方程组$A\begin{pmatrix}
    x \\ y
  \end{pmatrix}=\begin{pmatrix}
    u \\ v
  \end{pmatrix}$,此方程对任意$u,v\in\MR$有解当且仅当$\det A\ne0$,此时$\begin{pmatrix}
    x \\ y
  \end{pmatrix}=A^{-1}\begin{pmatrix}
    u \\ v
  \end{pmatrix}$.
\end{solution}

\begin{solution}
  \begin{inparaenum}[(a)]
    \item 见问题 \ref{problem2.37} 的解答.

    \item 如果$\det A=\pm1$,则$f_A$是满射(证明见问题 \ref{problem2.37} 的解答). 现在我们证明,如果$f_A$是满射,则$\det A=\pm1$. 由$f_A$是满射,存在$\begin{pmatrix}
          x_1 \\ y_1
        \end{pmatrix}\in\MM_{2,1}(\MZ)$,使得$f_A\begin{pmatrix}
          x_1 \\ y_1
        \end{pmatrix}=\begin{pmatrix}
          1 \\ 0
        \end{pmatrix}$;且存在$\begin{pmatrix}
          x_2 \\ y_2
        \end{pmatrix}\in\MM_{2,1}(\MZ)$,使得$f_A\begin{pmatrix}
          x_2 \\ y_2
        \end{pmatrix}=\begin{pmatrix}
          0 \\ 1
        \end{pmatrix}$. 于是
        \[
          A\begin{pmatrix}
            x_1 & x_2 \\
            y_1 & y_2
          \end{pmatrix} = \begin{pmatrix}
            1 & 0 \\
            0 & 1
          \end{pmatrix},
        \]
        所以$A$是可逆的,且$A^{-1}=\begin{pmatrix}
          x_1 & x_2 \\
          y_1 & y_2
        \end{pmatrix}\in\MM(\MZ)$. 由于$A,A^{-1}\in\MM_2(\MZ)$且$AA^{-1}=I_2$,我们有$\det A\det(A^{-1})=1$,这意味着$\det A=\pm1$.
  \end{inparaenum}
\end{solution}

\begin{solution}
  令$A=\begin{pmatrix}
    0 & 0 \\
    1 & 0
  \end{pmatrix}$,由于$A^n=O_2=O_2^n$,我们得到$f$不是单射.

  要证明$f$不是满射,令$B=\begin{pmatrix}
    0 & 1 \\
    0 & 0
  \end{pmatrix}$,我们证明方程$X^n=B$在$\MM_2(\MC)$中无解. 如果存在一个解$X\in\MM_2(\MC)$,则$\det X=0$,且由Cayley--Hamilton定理,我们有$X^2-tX=O_2$,其中$t=\Tr(X)$. 这意味着$X^n=t^{n-1}X\Rightarrow t^{n-1}X=B$. 在这个等式两边取迹,我们得到$t^n=0\Rightarrow X^2=O_2\Rightarrow X^n=O_2$,这与$X^n=B$矛盾.
\end{solution}

\begin{solution}
  \begin{inparaenum}[(a)]
    \item 令$g:\MR\to\MR,g(x)=x^{2016}+x^{2015}$,且令$Y=\begin{pmatrix}
          y & 0 \\
          0 & 0
        \end{pmatrix}$,其中$y<g\left(-\frac{2015}{2016}\right)$. 方程$f(X)=Y$在$\MM_2(\MR)$中无解.

    \item 令$Y=\begin{pmatrix}
      -1 & 0 \\
      0 & 0
    \end{pmatrix}$,方程$f(X)=Y$在$\MM_2(\MR)$中无解.
  \end{inparaenum}
\end{solution}

\begin{solution}
  \begin{inparaenum}[(a)]
    \item $AX=X'\Leftarrow X=A^{-1}X',A^{-1}=\frac1{ad-bc}
        \begin{pmatrix}
          d & b \\
          c & a
        \end{pmatrix}>0$. 于是$A^{-1}>0,X'>0$,且这意味着$A^{-1}X'>0$.

    \item 令$X_1,X_2$是$X$的列向量,$X_1',X_2'$是$X'$的列向量,我们有$X_1'=AX_1,X_2'=AX_2$. 如果$X_1=\begin{pmatrix}
          x \\ y
        \end{pmatrix}$,则$X_1'=\begin{pmatrix}
          ax - by \\
          -cx + dy
        \end{pmatrix}$,且条件$X>0,X'>0$说明
        \[
          \left\{
            \begin{aligned}
              & ax - by > 0 \\
              & -cx - dy > 0 \\
              & x > 0 \\
              & y > 0
            \end{aligned}
          \right..
        \]

      从几何的角度看,上面的每个不等式都表示一个半平面,我们需要证明这些半平面相交. 第一个半平面的边界是一条斜率为$m_1=\frac ab$的直线,第二个半平面的边界是一条斜率为$m_2=\frac cd$的直线. 如果我们有$m_1>m_2$,即$ad-bc>0$成立,则这两个半平面相交于第一象限. 这只需要取$X_1=X_2$,其中$x$和$y$是原先的不等式组的一个解.
  \end{inparaenum}
\end{solution}

\setcounter{solution}{42}

\begin{solution}
  由Cayley--Hamilton定理,我们有
  \[
    \left\{
      \begin{aligned}
        & (AB)^2 - \Tr(AB)AB + \det(AB)I_2 = O_2 \\
        & (BA)^2 - \Tr(BA)BA + \det(BA)I_2 = O_2
      \end{aligned}
    \right..
  \]
  由于$\Tr(AB)=\Tr(BA)$且$\det(AB)=\det(BA)$,我们得到$\Tr(AB)(AB-BA)=O_2$. 由于$\Tr(AB)>0$,这意味着$AB=BA$.
\end{solution}

\begin{solution}
  $\{I_2,A,A^2\}$.
\end{solution}

\begin{solution}
  设$n\ge2$. 要证明$\mathscr P_n\subseteq\mathscr P_2$,我们需要证明对任意$X\in\MM_2(\MC)$,存在$Y\in\MM_2(\MC)$使得$X^n=Y^2$. 令$J_X$是$X$的Jordan标准形,设$P$是一个可逆矩阵,使得$X=PJ_XP^{-1}$,且令$Y=PY_1P^{-1}$,方程$X^n=Y^2$变为$J_X^n=Y_1^2$. 下面我们分两种情形.

  如果$J_X=\begin{pmatrix}
    \lambda_1 & 0 \\
    0 & \lambda_2
  \end{pmatrix}$,则$J_X^n=\begin{pmatrix}
    \lambda_1^n & 0 \\
    0 & \lambda_2^n
  \end{pmatrix}$,且我们取$Y_1=\begin{pmatrix}
    \mu_1 & 0 \\
    0 & \mu_2
  \end{pmatrix}$,其中$\mu_1,\mu_2\in\MC$满足$\mu_1^2=\lambda_1^n$且$\mu_2^2=\lambda_2^n$.

  如果$J_X=\begin{pmatrix}
    \lambda & 1 \\
    0 & \lambda
  \end{pmatrix}$,则$J_X^n=\begin{pmatrix}
    \lambda^n & n\lambda^{n-1} \\
    0 & \lambda^n
  \end{pmatrix}$. 如果$Y_1=\begin{pmatrix}
    a & b \\
    0 & a
  \end{pmatrix}$,则我们有$Y_1^2=\begin{pmatrix}
    a^2 & 2ab \\
    0 & a^2
  \end{pmatrix}$,且我们得到方程$a^2=\lambda^n$与$2ab=n\lambda^{n-1}$. 如果$\lambda=0$,我们取$a=b=0$. 如果$\lambda\ne0$,则我们取$a\in\MC^\ast$使得$a^2=\lambda^n$且$b=\frac{n\lambda^{n-1}}{2a}$.

  要证明结论$\mathscr P_2\subseteq\mathscr P_n$,我们需要证明对任意$X\in\MM_2(\MC)$,存在$Y\in\MM_2(\MC)$,使得$X^2=Y^n$. 和前面结论的证明一样,化归到Jordan标准形,我们需要证明$J_X^2=Y_1^n$.

  如果$J_X=\begin{pmatrix}
    \lambda_1 & 0 \\
    0 & \lambda_2
  \end{pmatrix}$,我们取$Y_1=\begin{pmatrix}
    \mu_1 & 0 \\
    0 & \mu_2
  \end{pmatrix}$,其中$\mu_1^n=\lambda_1^2$且$\mu_2^n=\lambda_2^2$.

  如果$J_X=\begin{pmatrix}
    \lambda & 1\\
    0 & \lambda
  \end{pmatrix}$,则$J_X^2=\begin{pmatrix}
    \lambda^2 & 2\lambda \\
    0 & \lambda^2
  \end{pmatrix}$,我们取$Y_1=\begin{pmatrix}
    a & b \\
    0 & a
  \end{pmatrix}$,则$Y_1^n=\begin{pmatrix}
    a^n & na^{n-1}b \\
    0 & a^n
  \end{pmatrix}$. 我们得到方程$a^n=\lambda^n$与$na^{n-1}b=2\lambda$.

  如果$\lambda=0$,我们取$a=b=0$. 如果$\lambda\ne0$,我们取$a\in\MC^\ast$满足$a^n=\lambda^2$且$na^{n-1}b=2\lambda$.
\end{solution}

\begin{solution}
  \begin{inparaenum}[(a)]
    \item 由Cayley--Hamilton定理,我们有$A^2-\Tr(A)A+I_2=O_2$,于是$A+A^{-1}=\Tr(A)I_2$. 这意味着$AB+A^{-1}B=\Tr(A)B$,两边取迹,可得$\Tr(AB+A^{-1}B)=\Tr(AB)+\Tr(A^{-1}B)
        =\Tr\big(\Tr(A)B\big)=\Tr(A)\Tr(B)$.

    \item $B+B^{-1}=\Tr(B)I_2\Rightarrow BAB+BAB^{-1}=\Tr(B)BA$,等式两边取迹可得
        \[
          \Tr(BAB + BAB^{-1}) = \Tr\big( \Tr(B)BA \big) \Rightarrow \Tr(BAB) + \Tr(BAB^{-1}) = \Tr(B)\Tr(BA).
        \]

        由于$\Tr(BAB^{-1})=\Tr(A)$且$\Tr(BA)=\Tr(AB)$,我们得到$\Tr(BAB)+\Tr(A)=\Tr(B)\Tr(AB)$.
  \end{inparaenum}
\end{solution}

\begin{solution}
  \begin{inparaenum}[(a)]
    \item 通过反证法,我们假定存在$n\in\MN$使得$A^n=I_2$,这意味着$\det(A^n)=\det{}^nA=1$. 如果$\lambda_1,\lambda_2$是$A$的特征值,则$\lambda_1+\lambda_2=\Tr(A)>2,
        \lambda_1\lambda_2=\det A=1$,进一步可得$\lambda_1^n\lambda_2^n=1$且
        $\lambda_1^n+\lambda_2^n=\Tr(A^n)=\Tr(I_2)=2$,这意味着$\lambda_1^n=\lambda_2^n=1\Rightarrow
        |\lambda_1|=|\lambda_2|=1$. 我们有$2=|\lambda_1|+|\lambda_2|\ge|\lambda_1+\lambda_2|=\Tr(A)>2$,矛盾.

    \item 设$A=rB$,然后注意到这个问题就化归到 \ref{prob2.47a} 了.
  \end{inparaenum}
\end{solution}

\begin{solution}
  设$\lambda_1,\lambda_2$是$A$的特征值,即方程$x^2-\Tr(A)x+\det A=0$的解. 由于$\varDelta=\Tr^2(A)-4\det A<0$,我们得到$\lambda_1,\lambda_2\in\MC\backslash\MR$且
  $\lambda_2=\bar{\lambda_1}$. 另一方面,$\lambda_1\lambda_2=1$,这意味着$\lambda_{1,2}=
  \cos\alpha\pm\ii\sin\alpha$,且$\sin\alpha\ne0$. 我们有$|\Tr(A^n)|=|2\cos(n\alpha)|\le2$.

  矩阵
  \[
    B_n = \begin{pmatrix}
      \cos\frac\pi n & - \sin\frac\pi n\\
      \sin\frac\pi n & \cos \frac\pi n
    \end{pmatrix}
  \]
  满足题中的要求.
\end{solution}

\begin{solution}
  充分性是显然的,下面证明必要性,即如果$C$与$A$和$B$都可交换,则$C=O_2$. 如果$A$或$B$形如$\alpha I_2,\alpha\in\MC$,则结论显然. 于是我们假定$A$和$B$都不是$\alpha I_2,\alpha\in\MC$的形式. 由于$AC=CA$且$CB=BC$,由定理 \ref{thm1.1},我们有$C=\alpha_1A+\beta_1I_2$且$C=\alpha_2B+\beta_2I_2$对某个$\alpha_1,\alpha_2,\beta_1,\beta_2\in\MC$成立,于是$\alpha_1A+\beta_1I_2=\alpha_2B+\beta_2I_2$.

  如果$\alpha_1=0$,我们得到$\beta_1I_2=\alpha_2B+\beta_2I_2$. 如果$\alpha_2\ne0$,我们得到$B=\frac{\beta_1-\beta_2}{\alpha_2}I_2$,这是不可能的. 因此$\alpha_2=0$,且$\beta_1=\beta_2$. 由于$\alpha_1=0$,我们得到$C=\beta_1I_2$,且$\Tr(C)=0\Rightarrow\beta_1=0
  \Rightarrow C=O_2 $.

  如果$\alpha_1\ne0$,我们得到$A=\frac{\alpha_2}{\alpha_1}B+\frac{\beta_2-\beta_1}{\alpha_1}I_2=\delta B+\gamma I_2$,其中$\delta=\frac{\alpha_2}{\alpha_1}$且$\gamma=\frac{\beta_2-\beta_1}{\alpha_1}$. 于是$C=AB-BA=(\delta B+\gamma I_2)B-B(\delta B+\gamma I_2)=O_2$.
\end{solution}

\begin{solution}
  令$f_A(x)=\det(A-xI_2)=x^2-\Tr(A)x+\det A$是$A$的特征多项式,且$\lambda_1,\lambda_2$是它的根. 由于$A^n=I_2$,我们得到$\lambda_1^n=\lambda_2^n=1\Rightarrow
  |\lambda_1|=|\lambda_2|=1$,且条件$\det(A-I_2)\in\MR$意味着$f_A(1)=1-(\lambda_1+\lambda_2)+\lambda_1\lambda_2\in\MR$.

  我们有
  \begin{align*}
    \lambda_1\lambda_2 - (\lambda_1 + \lambda_2)
    \in \MR & \Leftrightarrow \bar{\lambda_1}\cdot\bar{\lambda_2} - (\bar{\lambda_1} + \bar{\lambda_2}) \in \MR \\
    & \Leftrightarrow \frac1{\lambda_1}\cdot\frac1{\lambda_2} - \frac1{\lambda_1} - \frac1{\lambda_2} \in \MR \\
    & \Leftrightarrow \frac{1-\lambda_1-\lambda_2}{\lambda_1\lambda_2} \in \MR.
  \end{align*}

  设$\lambda_1+\lambda_2=\lambda_1\lambda_2+a,a\in\MR$,且$\frac{1-\lambda_1\lambda_2-a}{\lambda_1\lambda_2}
  =b\in\MR$. 这些就意味着$\lambda_1\lambda_2\in\MR$且$\lambda_1+\lambda_2\in\MR$,因此$f_A\in\MR[x]$.
\end{solution}

\begin{solution}
  \begin{inparaenum}[(a)]
    \item 我们有$(AB)^n=O_2\Rightarrow (AB)^2=O_2\Rightarrow B(AB)^2A=O_2\Rightarrow(BA)^3=O_2
        \Rightarrow(BA)^2=O_2\Rightarrow(BA)^n=O_2$.

    \item 令$f=f_{AB}=f_{BA}$是$AB$和$BA$的特征多项式,我们有$x^n=Q(x)f(x)+ax+b$,其中$Q\in\MR[x]$且$a,b\in\MR$. 因此,$(AB)^n=I_2\Leftrightarrow Q(AB)f(AB)+aAB+bI_2=I_2\Leftrightarrow
        aAB+(b-1)I_2=O_2$.

        如果$a=0$,则$b=1$,且这意味着$(BA)^n=Q(BA)f(BA)+I_2=I_2$.

        如果$a\ne0$,则$AB=\frac{1-b}aI_2$,其中$\frac{1-b}a\ne0$,由于$A$与$B$是可逆的,这意味着$AB=BA$.

        \item $(AB)^n=C\Leftrightarrow aAB+bI_2=C$. 另一方面,$(BA)^n=C\Leftrightarrow aBA+bI_2=C$,由于$AB\ne BA$,我们得到$a=0$且$C=bI_2$.
  \end{inparaenum}
\end{solution}

\begin{solution}
  $\alpha AB+\beta BA=I_2\Rightarrow \alpha(AB-BA)=I_2-(\alpha+\beta)BA$且$\beta(BA-AB)=I_2
  -(\alpha+\beta)AB$. 且
  \begin{align*}
    \det(I_2 - xBA) & = \det (A^{-1}I_2A - A^{-1}xABA) \\
    & = \det(A^{-1}) \det (I_2 - xAB) \det A\\
    & = \det (I_2 - xAB).
  \end{align*}

  在前面的等式两边取行列式可得
  \[
    \alpha^2 \det(AB - BA) = \beta^2 \det(BA - AB)\quad \Leftrightarrow \quad
    (\alpha^2 - \beta^2)\det(AB - BA) = 0,
  \]
  由于$\alpha\ne\pm\beta$,这说明$\det(AB-BA)=0$.
\end{solution}

\begin{solution}
  令$f(x)=\det(A^2+B^2+C^2+xBC)\in\MR[x]$,我们有$f(-2)=\det\big(A^2+(B-C)^2\big)\ge0$,且$f(2)=
  \det\big(A^2+(B+C)^2\big)\ge0 $. 由推论 \ref{coro2.4},直接计算可得
  \[
    f(x) = \det(A^2 + B^2 + C^2) + \alpha x + x^2 \det B\det C = \det(A^2 + B^2 + C^2) + \alpha x
  \]
  对某个$\alpha\in\MR$成立. 因此,$f$是一个单调的次数为1的多项式,且由于0介于$-2$与2之间,我们得到$f(0)=\det(A^2 + B^2 + C^2)\ge0$.
\end{solution}

\setcounter{solution}{54}

\begin{solution}
  我们有$z^4-az^3-az+1=(z^2-\alpha_1z+1)(z^2-\alpha_2z+1)$,其中$\alpha_{1,2}=\frac{a\pm\sqrt{a^2+8}}2$. 由于$a\in(-1,1)$,我们得到$|\alpha_{1,2}|<2$,这意味着方程$z^2-\alpha_1z+1=0$和$z^2-\alpha_2z+1=0$只有复数解. 它们的解为$z_1,z_2$和$z_3,z_4$,且满足$z_1z_2=z_3z_4=1$. 由于$z_1,z_2$与$z_3,z_4$都是复共轭的,因此有$|z_1|=|z_2|=|z_3|=|z_4|=1$. 我们有
  \begin{align*}
    \det [ (A^2 - \alpha_1A + I_2)( A^2 - \alpha_2A + I_2 ) ] = 0 \Rightarrow {}& \det (A^2 - \alpha_1A + I_2) = 0 \\
    \text{或}{} & \det ( A^2 - \alpha_2A + I_2 ) = 0.
  \end{align*}

  如果
  \[
    \det (A^2 - \alpha_1A + I_2) = 0 \Leftrightarrow \det (A - z_1I_2)(A - z_2I_2) = |\det(A - z_1I_2)|^2 = 0,
  \]
  这意味着$\det(A-z_1I_2)=0$. 由推论 \ref{coro2.5},可得
  \[
    0 = \det(A - z_1I_2) = \det A - \Tr(A)z_1 + z_1^2 = \det A - 1 + z_1\big(\alpha_1 - \Tr(A)\big),
  \]
  这就说明$\det A=1$.
\end{solution}

\begin{solution}
  由于$B$是幂零的,我们可知它的特征值都是0. 由定理 \ref{thm2.1},可得$\lambda_{A+B}=\lambda_A+\lambda_B=\lambda_A$,且$\mu_{A+B}=\mu_A+\mu_B
  =\mu_A$. 这意味着$\det(A+B)=\lambda_{A+B}\mu_{A+B}=\lambda_A\mu_A=\det A$.
\end{solution}

\begin{solution}
  如果$m=1$或$n=1$,那么问题是平凡的,因此我们假定$m$与$n$都是不小于2的. 注意到$A^2=B^2=O_2$,利用问题 \ref{problem1.8} 即可.
\end{solution}

\begin{solution}
  首先我们证明,如果$AB$和$BA$都是幂零矩阵,则$A+B$也是幂零矩阵. 我们有
  \begin{gather*}
    (A + B)^2 = A^2 + B^2 + AB + BA = AB + BA, \\
    (A + B)^4 = (AB + BA)^2 = (AB)^2 + AB^2A + BA^2B + (BA)^2 = O_2.
  \end{gather*}
  由引理 \ref{lemma2.3},这就说明$(A+B)^2=O_2$.

  现在我们证明,如果$A+B$是幂零矩阵,则$AB$与$BA$都是幂零矩阵. 我们有
  \[
    (A + B)^2 = O_2 \Rightarrow A^2 + AB + BA + B^2 = O_2 \Rightarrow AB = -BA.
  \]
  这意味着
  \begin{gather*}
    (AB)^2 = ABAB = A(-AB)B = -A^2B^2 = O_2, \\
    (BA)^2 = BABA = B(-BA)A = -B^2A^2 = O_2.
  \end{gather*}
\end{solution}

\begin{solution}
  我们对矩阵$AB-BA$应用Cayley--Hamilton定理,可得
  \[
    (AB - BA)^2 - \Tr(AB - BA)(AB - BA) + \det(AB - BA)I_2 = O_2.
  \]
  由于$\Tr(AB - BA)=0$,那么$(AB - BA)^2+\det(AB - BA)I_2=O_2$. 这意味着$\det(AB-BA)=0\Leftrightarrow(AB-BA)^2=O_2$. 由引理 \ref{lemma2.10},我们有$\det(AB-BA)=0\Leftrightarrow\Tr(A^2B^2)
  =\Tr\big((AB)^2\big) $.
\end{solution}

\setcounter{solution}{60}

\begin{solution}
  我们有
  \[
    (AB)^2 = AB^2A \Rightarrow \Tr\big((AB)^2\big)
    = \Tr(AB^2A) = \Tr(A^2B^2).
  \]
  且由问题 \ref{problem2.59},我们得到$(AB-BA)^2=O_2$. 这意味着
  \[
    (AB)^2 - AB^2A + (BA)^2 - BA^2B = O_2.
  \]
  由于$(AB)^2=AB^2A$,我们得到$(BA)^2=BA^2B$. 充分性也可以用同样的方法证明.
\end{solution}

\begin{solution}
  由引理 \ref{lemma2.11},我们有
  \[
    \det (A - B) \det (A + B) = \det (A^2 - B^2)
    \quad \Leftrightarrow \quad
    \det (AB - BA) = 0.
  \]
  然而,$(AB-BA)^2=-\det(AB-BA)I_2$,我们有$(AB-BA)^2=O_2\Leftrightarrow\det(AB-BA)=0$.
\end{solution}

\begin{solution}
  利用引理 \ref{lemma2.12}.
\end{solution}

\begin{solution}
  如果$\det(AB-BA)\ge0$,由引理 \ref{lemma2.12},我们有$\det(A^2+B^2)\ge0$.

  如果$\det(AB-BA)\le0$,则$\det(AB+BA)\le0$. 由推论 \ref{coro2.1},我们有
  \begin{align*}
    \det[(A^2 + B^2) + (AB + BA)] +& \det [(A^2 + B^2) - (AB + BA)] \\
    & = 2\det(A^2 + B^2) + 2\det(AB + BA).
  \end{align*}
  然而,$\det(A^2+B^2+AB+BA)=\det[(A+B)^2]=\det{}^2(A+B)$,且
  \[
    \det [(A^2 + B^2) - (AB + BA)] = \det{}^2 (A - B).
  \]
  于是可得
  \[
    \det{}^2 (A + B) + \det{}^2(A - B) = 2\det(A^2 + B^2) + 2\det(AB + BA).
  \]
  由于$\det(AB+BA)\le0$,我们有$\det(A^2+B^2)\ge0$.
\end{solution}

\begin{solution}
  由问题 \ref{problem1.32},我们有
  \begin{align*}
    \det(A^n + B^n + AB) = {}& \det (A^n + B^n) + \det(A^n + AB) + \det(B^n + AB) \\
    & -\det(A^n) - \det(B^n) - \det(AB) \\
    = {}& \det(A^n + B^n) + \det A\det(A^{n-1} + B) + \det(B^{n-1} + A)\det B \\
    & - \det(A^n) - \det(B^n) - \det(BA) \\
    = {}& \det(A^n + B^n) + \det(A^{n-1} + B)\det A + \det B\det(B^{n-1} + A) \\
    & - \det(A^n) - \det(B^n) - \det(BA) \\
    = {}& \det (A^n + B^n) + \det(A^n + BA) + \det(B^n + BA) \\
    & - \det(A^n) - \det (B^n) - \det(BA) \\
    = {}& \det(A^n + B^n + BA).
  \end{align*}

  类似地,我们可以证明$\det(A^n+B^n-AB)=\det(A^n+B^n-BA)$.
\end{solution}

\begin{solution}
  由问题 \ref{problem2.65},当$n=2$时,我们有
  \[
    \det (A^2 + B^2 - AB) = \det(A^2 + B^2 - BA).
  \]
  由于$A^2+B^2-AB=O_2$且$A^2+B^2-BA=AB-BA$,我们得到$\det(AB-BA)=0\Leftrightarrow(AB-BA)^2=O_2$.
\end{solution}

\begin{solution}
  由于$A^2=O_2$,我们得到$\det A=0$. 由引理 \ref{lemma2.12},我们有
  \[
    \det(B^2) = \det(AB - BA) + (\det B)^2 + [\det(A + B) - \det B]^2,
  \]
  这意味着$0=\det(AB-BA)+[\det(A+B)-\det B]^2$,那么显然有$\det(AB-BA)=0\Leftrightarrow\det(A+B)=\det B$.
\end{solution}

\begin{solution}
  由于$A^2=O_2$,我们得到$\det A=0$,且$\det(AB)=0,\forall B\in\MM_2(\MR)$. 由引理 \ref{lemma2.7},我们得到
  \begin{align*}
    \det(AB - BA) & = \det (AB) - \Tr(AB)\Tr(BA) + \Tr(AB^2A) + \det (BA) \\
    & = 2\det(AB) - \Tr^2(AB) + \Tr(A^2B^2) \\
    & = -\Tr^2(AB) \\
    & \le 0.
  \end{align*}
  类似地,我们也可以证明$\det(AB+BA)=\Tr^2(AB)\ge0$.
\end{solution}

\setcounter{solution}{70}

\begin{solution}
  设$A\in\MM_2(\MC)$满足$A^2=O_2$,且令$X\in\MM_2(\MC)$满足$AX=XA$,则$\det(AX-XA)=0$. 且由问题 \ref{problem2.67},我们有$\det(A+X)=\det X$.

  要证明充分性,我们设$x_1,x_2$是方程$\det(A+xI_2)=x^2+\Tr(A)x+\det A=0$的解. 由于$x_1I_2,x_2I_2\in\mathscr C(A)$,我们得到$0=|\det(A+xI_2)|\ge|x_i|^2,i=1,2$,这意味着$x_1=x_2=0$. 因此,$\Tr(A)=\det A=0\Rightarrow A^2=O_2$.
\end{solution}

\begin{solution}
  \begin{inparaenum}[(a)]
    \item $(A-B)(A^{-1}-B^{-1}) = I_2\Rightarrow AB^{-1} + BA^{-1}
    =I_2 \Rightarrow A-B=AB^{-1}A$ 且 $B-A=BA^{-1}B$. 于是
    \[
      \frac{(\det B)^2}{\det A} = \det(BA^{-1}B) =
      \det(B - A) = \det(A - B) = \det(AB^{-1}A) =
      \frac{(\det A)^2}{\det B},
    \]
    这说明$\det A=\det B$且$\det(A-B)=\det A$.

    我们给出以下两个满足题中条件的矩阵$A,B\in\MM_2(\MR)$. 设$\alpha,
    u,x\in\MR$,且$\alpha,x\ne0$,令
    \[
      A = \begin{pmatrix}
        x & \frac{\alpha x- \alpha^2 - x^2} u \\
        u & \alpha - x
      \end{pmatrix}  \quad \text{且}\quad
      B = \begin{pmatrix}
        x - \alpha & \frac{\alpha - \alpha^2 - x^2} u \\
        u & - x
      \end{pmatrix}.
    \]
    则
    \[
      A^{-1} = \frac1{\alpha^2} \begin{pmatrix}
        \alpha - x & \frac{x^2 + \alpha^2-\alpha x}u \\
        -u & x
      \end{pmatrix} \quad \text{且} \quad
      B^{-1} = \frac1{\alpha^2} \begin{pmatrix}
        -x & \frac{x^2 + \alpha^2-\alpha x}u \\
        u & -x
      \end{pmatrix}.
    \]
    我们有$\det A=\det B = \det(A-B)=\alpha^2$.

    \item 此时结论不成立. 令$\varepsilon=\frac{1+\ii\sqrt3}2$,再令
        \[
          A = \begin{pmatrix}
            \varepsilon & 0 \\
            0 & 1
          \end{pmatrix},\quad B = \begin{pmatrix}
            1 & 0 \\
            0 & \bar\varepsilon
          \end{pmatrix},
        \]
        则$A^{-1}-B^{-1}=(A-B)^{-1}$,但$\det A\ne \det B$.
  \end{inparaenum}
\end{solution}

\begin{solution}
  只需要考虑$P$是一个二次多项式即可.
\end{solution}

\setcounter{solution}{74}

\begin{solution}
  如果$A$或$B$是可逆的,则$B=O_2$或$A=O_2$,结论显然.

  现在我们假定$A$和$B$都不可逆. 由于$AB=O_2$,我们得到
  \[
    (A + B)^n = A^n + B^n + B(A^{n-2} + BA^{n-3} + \cdots + B^{n-2})A = A^n + B^n + C.
  \]
  所有包含$AB$的式子都等于$O_2$. 由问题 \ref{problem1.32},我们有
  \begin{align*}
    \det (A + B)^n = {}& \det (A^n + B^n + C^n) \\
    = {}& \det (A^n + B^n) + \det (A^n + C) + \det (B^n + C) \\
    & - \det (A^n) - \det (B^n) - \det C \\
    = {}& \det(A^n + B^n),
  \end{align*}
  这是因为$\det(A^n)=\det(B^n)=\det C=\det(A^n+C)
  =\det(B^n+C)=0$.
\end{solution}

\begin{solution}
  \begin{inparaenum}[(a)]
    \item 令$A=I_2,B=\begin{pmatrix}
      0 & 1 \\
      -1 & 0
    \end{pmatrix}$,则$\det(xA+yB)=\begin{vmatrix}
      x & y \\
      -y & x
    \end{vmatrix}=x^2+y^2$.

    \item 对任意矩阵$A,B,C\in\MM_2(\MR)$,我们取不全为零的实数$x_0,y_0,z_0$,使得矩阵$x_0A+y_0B+z_0C$的第一行为0. 如果矩阵$A,B,C$的第一行分别为$[a_1,a_2],[b_1,b_2],[c_1,c_2]$,则方程组
    \[
      \left\{
        \begin{aligned}
          & a_1x + b_1y + c_1z = 0\\
          & a_2x + b_2y + c_2z = 0
        \end{aligned}
      \right.
    \]
    有非平凡解$(x_0,y_0,z_0)\ne(0,0,0)$,则$\det(x_0A+y_0B+z_0C)=0$,且
  $x_0^2+y_0^2+z_0^2\ne0$.
  \end{inparaenum}
\end{solution}

\section{Quickies}
\begin{problem}
  设$A,B\in\MM_2(\MC)$满足$ABAB=O_2$,是否有$BABA=O_2$?
\end{problem}

\begin{problem}
  设矩阵$A\in\MM_2(\MR)$满足$A^k\ne\lambda I_2,k\in\MN$. 证明:如果矩阵$A^k$的$(1,2)$元等于0,则所有形如$A^n,n\in\MN$的矩阵的$(1,2)$元等于0.
\end{problem}

\begin{problem}
  是否存在矩阵$A,B\in\MM_2(\MZ)$,使得$\det(A+2B)=3$且$\det(A+5B)=7$.
\end{problem}

\begin{problem}
  设$A,B\in\MM_2(\MQ)$满足$\det A=0$且$\det(A+\sqrt2B)=2$,证明:$\det B=1$且$\det(A+\sqrt pB)=p$对任意素数$p$成立
\end{problem}

\begin{problem}
  设$A,B\in\MM_2(\MQ)$满足$\det A=1$且$\det(A+\sqrt3B)=4$,证明:$\det B=1$且$\det(A+\sqrt5B)=6$.
\end{problem}

\begin{problem}
  设$A,B\in\MM_2(\MQ)$满足$\det(A+\sqrt[3]3B)=3$,证明:$\det(A+\sqrt2B)=3$.
\end{problem}

\begin{problem}
  设$A,B\in\MM_2(\MC)$满足$\Tr(AB)=0$,证明:$(AB)^2=(BA)^2$.
\end{problem}

\begin{problem}
  设$A,B\in\MM_2(\MR)$满足$A^2+B^2+2AB=O_2$且$\det(A^2-B^2)=0$,证明:$\det\big(\Tr(A)A-\Tr(B)B\big)=0$.
\end{problem}

\begin{problem}
  设$n\in\MN$,证明:如果$A\in\MM_2(\MC)$满足$\Tr(A)=0$,则$\Tr(A^{2n+1})=0$.
\end{problem}

\begin{problem}
  设$A\in\MM_2(\MC)$满足$A^k=A^{k+1}$对某个正整数$k$成立,证明:
  \[
    \Tr(A) = \Tr(A^2) = \Tr(A^3) = \cdots = \Tr(A^n) = \cdots.
  \]
\end{problem}

\begin{problem}
  设$A\in\MM_2(\MC)$满足$\Tr(A^n)=\Tr(A^{n+1})$对某个$n\in\MN$成立,证明:$A^2=O_2$.
\end{problem}

\begin{problem}
  $A,B\in\MM_2(\MC)$有相同的特征多项式(因此也就具有相同的特征值)当且仅当$\Tr(A^k)=\Tr(B^k)$对$k=1,2$成立,由此推断$A$是幂零矩阵当且仅当$\Tr(A^k)=0$对$k=1,2$成立.
\end{problem}

\begin{mybox}
  \begin{problem}[Jacobson引理.]

    设$A,B\in\MM_2(\MC)$,且$C=AB-BA$,证明:如果$C$与$A$或$B$可交换,则$C$是幂零矩阵.
  \end{problem}
\end{mybox}

\begin{problem}[\kaishu 矩阵是否相等?]

  设$A,B\in\MM_2(\MC)$满足$\Tr(A)=\Tr(B)$且$\det A=\det B$,能否得出$A=B$?
\end{problem}

\begin{problem}
  设$A,B\in\MM_2(\MC)$满足$\Tr(A)=\Tr(B)$,则$A(A-B)B=B(A-B)A$.
\end{problem}

\begin{mybox}
  \begin{problem}
    设$A\in\MM_2(\MC)$满足$A^k=O_2,k\in\MN$,证明:$(I_2-A)^{-1}=I_2+A$.
  \end{problem}
\end{mybox}

\begin{problem}
  对任意$A,B\in\MM_2(\MR)$,存在$\alpha\in\MR$,使得$(AB-BA)^2=\alpha I_2$.
\end{problem}

\begin{problem}
  \begin{enuma}
    \item 找出两个矩阵$A,B$,使得$A^2+B^2=\begin{pmatrix}
      1 & 2 \\
      2 & 1
    \end{pmatrix}$.

    \item 证明:任意两个满足$A^2+B^2=\begin{pmatrix}
      1 & 2 \\
      2 & 1
    \end{pmatrix}$的矩阵都是不可交换的.
  \end{enuma}
\end{problem}

\section{解答}

\begin{solution}
  答案是肯定的,因为$ABAB=O_2\Rightarrow BABABA=O_2
  \Rightarrow(BA)^3=O_2\Rightarrow(BA)^2=O_2$.
\end{solution}

\begin{solution}
  由定理 \ref{thm3.2},可知存在数列$(x_n)_{n\in\MN}$与$(y_n)_{n\in\MN}$,使得$A^n=x_nA+y_nI_2$对任意$n\in\MN$成立. 当$n=k$时,我们有$A^k=x_kA+y_kI_2$,这意味着$A=\frac1{x_k}(A^k-y_kI_2)=\begin{pmatrix}
    \ast & 0 \\
    \ast & \ast
  \end{pmatrix}$. 因此,如果$n\in\MN$,我们得到$A^n=x_nA+y_nI_2=\begin{pmatrix}
    \ast & 0 \\
    \ast & \ast
  \end{pmatrix}$.
\end{solution}

\begin{solution}
  这样的矩阵是不存在的. 由引理 \ref{lemma2.7},我们有$\det(A+xB)=\det A+\alpha x+x^2\det B$,其中$\alpha\in\MZ$. 如果$k\in\MZ$,我们有$\det(A+kB)=\det A+mk$对某个$m\in\MZ$成立. 因此,
  \[
    7 = \det(A + 5B) = \det[ (A + 2B) + 3B ]
    = \det (A + 2B) + 3m' = 3(1 + m')
  \]
  对某个$m'\in\MZ$成立,这是不可能的.
\end{solution}

\begin{solution}
  设$f(x)=\det(A+xB)=\alpha x+x^2\det B$,其中$
  \alpha\in\MQ$且$\det B\in\MQ$. 由于$f(\sqrt2)=2$,我们得到$\alpha+\sqrt2\det B=\sqrt2$,这意味着$\alpha=0$且$\det B=1$. 于是,$\det(A+\sqrt pB)=f(\sqrt p)=p\det B=p$.
\end{solution}

\begin{solution}
  设$f(x)=\det(A+xB)=1+\alpha x+x^2\det B\in\MQ[x],\alpha\in\MQ$. 由于$f(\sqrt3)=4$,我们得到$1+\alpha\sqrt3+3\det B=4$,由于$\det B\in\MQ$,这意味着$\alpha=0$且$\det B=1$. 因此,$f(x)=1+x^2$且$\det(A+\sqrt5B)=f(\sqrt5)=6$.
\end{solution}

\begin{solution}
  设$f(x)=\det(A+xB)=\det A+\alpha x+x^2\det B\in\MQ[x],\alpha\in\MQ$. 由于$f(\sqrt[3]3)=3$,我们得到$\det A+\alpha\sqrt[3]3+\sqrt[3]9\det B=3$,于是$\det A=3$且$\alpha=\det B=0$. 因此,$f(x)=3$且$\det(A+\sqrt2B)=f(\sqrt2)=3$.
\end{solution}

\begin{solution}
  由于$\Tr(AB)=\Tr(BA)=0$,由Cayley--Hamilton定理,我们得到$(AB)^2=-\det(AB)I_2=-\det(BA)I_2=(BA)^2$.
\end{solution}

\setcounter{solution}{84}

\begin{solution}
  设$\lambda_1,\lambda_2$是$A$的特征值,$A^{2n+1}$的特征值为$\lambda_1^{2n+1},\lambda_2^{2n+1}$. 由于$\Tr(A)=\lambda_1+\lambda_2=0$,我们有$\Tr(A^{2n+1})=\lambda_1^{2n+1}+\lambda_2^{2n+1}=-\lambda_2^{2n+1}
  +\lambda_2^{2n+1}=0$.
\end{solution}

\setcounter{solution}{86}

\begin{solution}
  设$\lambda_1,\lambda_2$是$A$的特征值,我们有$\Tr(A^n)=\lambda_1^n+\lambda_2^n=0$且$\Tr(A^{n+1})=\lambda_1^{n+1}+\lambda_2^{n+1}=0$. 由此可得$\lambda_1=\lambda_2=0$,且由定理 \ref{thm2.2},可知$A^2=O_2$.
\end{solution}

\begin{solution}
  显然只需要证明充分性即可. 设$\lambda_1,\lambda_2$是$A$的特征值,$\mu_1,\mu_2$是$B$的特征值. 由
  \[
    \left\{
      \begin{aligned}
        & \lambda_1 + \lambda_2 = \mu_1 + \mu_2 \\
        & \lambda_1^2 + \lambda_2^2 = \mu_1^2 + \mu_2^2
      \end{aligned}
    \right.
  \]
  计算可得$\lambda_1=\mu_1,\lambda_2=\mu_2$或
  $\lambda_1=\mu_2,\lambda_2=\mu_1$. 在这两种情形下,都有$A$和$B$具有相同的特征多项式.

  要证明问题的第二部分,只需要注意到$A$是幂零矩阵,当且仅当$A^2=O_2$,当且仅当$A$与$O_2$具有相同的特征多项式.
\end{solution}

\begin{solution}
  我们假定$C=AB-BA$与$A$可交换,现在来证明$C$是幂零矩阵. 我们有
  \begin{gather*}
    \Tr(C)  = \Tr(AB - BA) = 0, \\
    \Tr(C^2)  = \Tr\big( C(AB - BA) \big) = \Tr(CAB) - \Tr(CBA) = \Tr(ACB) - \Tr(CBA) = 0.
  \end{gather*}
  于是由问题 \ref{problem2.88},可知$C$是幂零矩阵.
\end{solution}

\begin{solution}
  这两个矩阵不一定相等,例如$A=\begin{pmatrix}
    2 & 3\\
    4 & 6
  \end{pmatrix},B=\begin{pmatrix}
    4 & 16 \\
    1 & 4
  \end{pmatrix}$.
\end{solution}

\begin{solution}
  定理 \ref{thm2.2} 说明$A^2 = \big(\Tr(A)\big)A - (\det A)I_2$且$B^2= \big(\Tr(B)\big)B - (\det B)I_2$. 由于$\Tr(A)=\Tr(B)$,我们有
  \begin{align*}
    A(A-B)B & = A^2B - AB^2 \\
    & = \big[ \big(\Tr(A)\big)A - (\det A)I_2\big]B - A\big[ \big(\Tr(B)\big)B - (\det B)I_2\big] \\
    & = A\det B - B\det A.
  \end{align*}
  类似地,我们可以证明$B(A-B)A=A\det B-B\det A$.
\end{solution}

\begin{solution}
  如果$k=1$,我们得到$A=O_2$,结论显然. 设$k\ge2$,如果$\lambda$是$A$的一个特征值,我们得到$\lambda^k=0$,意味着$\lambda=0$. 因此,$A$的所有特征值均为0. 那么由定理 \ref{thm2.2},这说明$A^2=O_2$. 我们有
  \[
    (I_2 - A)(I_2 + A) = I_2 - A^2 = I_2 \Rightarrow (I_2 - A)^{-1} = I_2 + A.
  \]
\end{solution}

\begin{solution}
  设$X=AB-BA$,由于$\Tr(X)=0$,根据定理 \ref{thm2.2},我们有$X^2=\alpha I$,其中$\alpha=-\det(AB-BA)$.
\end{solution}

\begin{solution}
  \begin{inparaenum}[(a)]
    \item 取$A=\begin{pmatrix}
      1 & 1 \\
      1 & 1
    \end{pmatrix},B=\begin{pmatrix}
      0 & 1 \\
      -1 & 0
    \end{pmatrix}$.

  \item 如果两个矩阵可交换,则$\det(A^2+B^2)=|\det(A+\ii B)|^2\ge0$,这与$\det(A^2+B^2)=\begin{vmatrix}
        1 & 2\\
        2 & 1
      \end{vmatrix}=-3<0$矛盾.
  \end{inparaenum}
\end{solution}




