\usepackage{etex,tabularx}
\newcolumntype{Y}{>{\centering\arraybackslash}X}
\newcolumntype{Z}{>{\centering\arraybackslash $}X<{$}}
\usepackage{fontspec}
\IfFontExistsTF{FZShuSong-Z01}{
  \PassOptionsToPackage{fontset=founder}{ctex}
}{}
\usepackage{xeCJK}
\defaultCJKfontfeatures{Mapping=fullwidth-stop}
\usepackage[heading]{ctex}
\usepackage{manfnt}
\usepackage{indentfirst}
\usepackage{zhnumber}
\usepackage[T1]{fontenc}
\renewcommand\rmdefault{ptm}
\usepackage[centering,
           top=2.54cm,bottom=2.54cm,right=2.9cm,left=2.9cm,
           headsep=25pt,headheight=20pt]{geometry}
\IfFileExists{mtpro2.sty}{
  \usepackage[zswash,amsbb,straightbraces]{mtpro2}
}{\usepackage{amssymb}\let\SQRT\sqrt}

\usepackage{mathtools,mathrsfs}
\usepackage{caption}

\usepackage{extarrows}
\usepackage{graphicx}

\setmainfont{Times New Roman}

\usepackage{pifont}
\usepackage{imakeidx}
\makeindex[
  title = {名词索引},
  intoc = true,
  columns = 2,
  columnsep = 1cm,
  columnseprule = true,
  program = makeindex,
  options = {-s mkind.ist},
  noautomatic = false
]
\indexsetup{
  toclevel = chapter,
  headers = {名词索引}{名词索引},
  othercode = {
    \renewcommand{\indexspace}{\smallskip}
  }
}

\usepackage[hyperindex]{hyperref}
\hypersetup{
  bookmarksopen = true ,
  bookmarksopenlevel = 1 ,
  bookmarksnumbered = true ,
  pdftitle = {二阶矩阵} ,
  pdfauthor = {Vasile Pop \& Ovidiu Furdui(向禹翻译)} ,
  linktoc = all ,
  CJKbookmarks = true,unicode,
  colorlinks ,
  linkcolor = blue ,
  citecolor = blue ,
  urlcolor = blue ,
  anchorcolor = blue
  }
\usepackage[Symbol]{upgreek}
\renewcommand{\pi}{\uppi}

\DeclareSymbolFont{ugmL}{OMX}{yhex}{m}{n}
\DeclareMathAccent{\wideparen}{\mathord}{ugmL}{"F3}

\IfFontExistsTF{FZShuSong-Z01}{
  \setCJKmainfont[BoldFont={FZHei-B01},ItalicFont={FZKai-Z03}]{FZShuSong-Z01}
}{}
\IfFontExistsTF{Microsoft YaHei}{
  \newCJKfontfamily\wryh{Microsoft YaHei}
}{
  \let\wryh\sffamily
}
\IfFontExistsTF{汉仪大宋简}{
  \newCJKfontfamily\hyds{汉仪大宋简}
}{
  \let\hyds\rmfamily
}

\defaultfontfeatures{Mapping=tex-text}




%\everymath{\displaystyle}
\lineskiplimit=2.5pt
\lineskip=2.5pt plus .5pt

\renewcommand{\le}{\leqslant}
\renewcommand{\ge}{\geqslant}

\allowdisplaybreaks[4]


\newcommand\OR{\overrightarrow}



\usepackage{fancyhdr}
\usepackage{tkz-euclide}
\usetikzlibrary{arrows.meta,decorations.fractals}
\tikzset{>=Stealth,samples=200}
%\renewcommand{\Re}{\operatorname{Re}}
%\renewcommand{\Im}{\operatorname{Im}}
\newcommand{\ii}{\mathrm i}
\ctexset{punct=kaiming}
\renewcommand\thempfootnote{\ding{45}}

\usepackage{tocloft}
\renewcommand\cftchapfont{\hyds}
\renewcommand{\cftchapleader}{\cftdotfill{\cftdotsep}}
\ctexset {
  chapter = {
  beforeskip = 0pt,
  fixskip = true,
  format = \Huge\hyds,
  nameformat = \rule{\linewidth}{1bp}\par\bigskip\hfill\label{chap\thechapter}\chapternamebox,
  number = \arabic{chapter},
  aftername = \par\medskip,
  aftertitle = \par\bigskip\nointerlineskip\rule{\linewidth}{2bp}\par
},
  section = {
    titleformat+  = \hyds\label{sec\thesection}\raggedright
  },
  subsection = {
    titleformat+  = \hyds
  },
  subsubsection = {
    titleformat+  = \hyds
  },
  contentsname = \hyds{目 录}
}
\newcommand\chapternamebox[1]{%
\parbox{\ccwd}{\linespread{1}\selectfont\centering #1}}


\usepackage{paralist}
\usepackage[inline]{enumitem}
\newenvironment{enum}{\begin{enumerate}[label=(\alph*),left=-0.1cm]}
{
\end{enumerate}}
\DeclareMathOperator{\ee}{\!\!\;\mathrm e}

\newcommand{\MR}{\mathbb R}
\newcommand{\MC}{\mathbb C}
\newcommand{\MF}{\mathbb F}
\newcommand{\MZ}{\mathbb Z}
\newcommand{\MN}{\mathbb N}
\newcommand{\MQ}{\mathbb Q}
\newcommand{\MCF}{\mathscr F}
\newcommand{\MM}{\mathscr M}

\newenvironment{eenum}{\begin{enumerate}
[label=(\arabic*)]}
{
\end{enumerate}}
\newenvironment{enuma}{\begin{enumerate*}[label=(\alph*),listparindent=2em,itemjoin=\\\hspace*{2em}]}
{
\end{enumerate*}}
\newcommand\quan[1]{
\tikz[baseline=(a.base)]\node(a)[inner sep=0.5pt,draw,circle]{$#1$};
}
\newcommand\closure[1]{%
{}\mkern2mu\overline{\mkern-2mu#1}
}
\renewcommand\bar{\closure}



\newcounter{definition}
\counterwithin{definition}{chapter}
\renewcommand\thedefinition{\thechapter.\arabic{definition}}
\newcommand{\defname}{定义}
\newenvironment{definition}[1][]{\par%
  \refstepcounter{definition}\label{def\thedefinition}%
  \noindent\textbf{\defname\thedefinition}\quad\textbf{#1}
}{\par}

\newcounter{thm}
\counterwithin{thm}{chapter}
\renewcommand\thethm{\thechapter.\arabic{thm}}
\newcommand{\theoremname}{定理}
\newenvironment{theorem}[1][]{\par%
  \refstepcounter{thm}\label{thm\thethm}%
  \noindent\textbf{\theoremname\thethm}\quad\textbf{#1}
  \parindent=2em\kaishu
}{\par}

\newcounter{prop}
\counterwithin{prop}{chapter}
\renewcommand\theprop{\thechapter.\arabic{prop}}
\newcommand{\propname}{命题}
\newenvironment{prop}{\par%
  \refstepcounter{prop}\label{prop\theprop}%
  \noindent\textbf{\propname\theprop}\quad
}{\par}

\newcounter{example}
\counterwithin{example}{chapter}
\renewcommand\theexample{\thechapter.\arabic{example}}
\newcommand{\examplename}{例}
\newenvironment{example}[1][]{\par%
  \refstepcounter{example}\label{exam\theexample}%
  \noindent\textbf{\examplename\theexample}\quad\textbf{#1}
}{\par}

\newcounter{lemma}
\counterwithin{lemma}{chapter}
\renewcommand\thelemma{\thechapter.\arabic{lemma}}
\newcommand{\lemmaname}{引理}
\newenvironment{lemma}[1][]{\par%
  \refstepcounter{lemma}\label{lemma\thelemma}%
  \noindent\textbf{\lemmaname\thelemma}\quad\textbf{#1}
  \parindent=2em\kaishu
}{\par}

\newcounter{coro}
\counterwithin{coro}{chapter}
\renewcommand\thecoro{\thechapter.\arabic{coro}}
\newcommand{\corname}{推论}
\newenvironment{corollary}[1][]{\par%
  \refstepcounter{coro}\label{coro\thecoro}%
  \noindent\textbf{\corname\thecoro}\quad\textbf{#1}
  \parindent=2em\kaishu
}{\par}

\newcounter{remark}
\counterwithin{remark}{chapter}
\renewcommand\theremark{\thechapter.\arabic{remark}}
\newcommand{\remarkname}{注}
\newenvironment{remark}[1][]{\par%
  \refstepcounter{remark}\label{remark\theremark}%
  \noindent\emph{\remarkname\theremark}\quad\textbf{#1}
}{\par}

\newcounter{property}
\counterwithin{property}{chapter}
\newcommand{\propertyname}{性质}
\newenvironment{property}{\par%
  \refstepcounter{property}\label{property\theproperty}%
  \noindent\emph{\propertyname\theproperty}\quad
}{\par}

\newenvironment{nota}{\par%
  \indent \textbf{注意.}
}{\par}

\newenvironment{proverb}{\par%
  \leftskip=4cm\rightskip=2cm
}{\par}

\newcounter{problem}
\counterwithin{problem}{chapter}
\renewcommand\theproblem{\thechapter.\arabic{problem}}
\newenvironment{problem}[1][]{\par%
  \refstepcounter{problem}\label{problem\theproblem}%
  \noindent\hyperlink{solution\thechapter-\arabic{problem}}{\textbf{\theproblem}}
  \hypertarget{problem\thechapter-\arabic{problem}}{}\textbf{#1}
}{\par}

\newcounter{solution}
\counterwithin{solution}{chapter}
\renewcommand\thesolution{\thechapter.\arabic{solution}}
\newenvironment{solution}{\par%
  \refstepcounter{solution}\label{solution\thesolution}%
  \noindent\hyperlink{problem\thechapter-\arabic{solution}}{\textbf{\thesolution}}
  \hypertarget{solution\thechapter-\arabic{solution}}{}
}{\par}



\usepackage[amsmath,thmmarks]{ntheorem}
{
\theoremstyle{nonumberplain}
\theoremheaderfont{\bfseries}
\theorembodyfont{\normalfont}
\theoremsymbol{\mbox{$\Box$}}
\newtheorem{proof}{\noindent 证\hspace*{0.5em}}
\newtheorem{solve}{\noindent 解\hspace*{0.5em}}
}

\setlist{itemsep=0pt}



\newcommand\pp[2]{\frac{\partial #1}{\partial #2}}
\newcommand\ppp[2]{\frac{\partial^2 #1}{\partial #2^2}}
\newcommand\pppp[3]{\frac{\partial^2 #1}{\partial #2\partial #3}}
\newcommand\dd[2]{\frac{\mathop{}\!\mathrm d#1}{\mathop{}\!\mathrm d#2}}
\newcommand\ddd[2]{\frac{\mathop{}\!\mathrm d^2#1}{\mathop{}\!\mathrm d#2^2}}
\newcommand\dx{\mathop{}\!\mathrm dx}
\newcommand\dy{\mathop{}\!\mathrm dy}
\newcommand\dz{\mathop{}\!\mathrm dz}
\newcommand\dif{\mathop{}\!\mathrm d}
\newcommand\TT{^{\mathrm T}}

\usepackage[subrefformat=parens]{subcaption}
\DeclareMathOperator{\Arg}{Arg}
\DeclareMathOperator{\GL}{GL}
\DeclareMathOperator{\SL}{SL}
\DeclareMathOperator{\Tr}{Tr}
\DeclareMathOperator{\Spec}{Spec}
\DeclareMathOperator{\Ker}{Ker}
\DeclareMathOperator{\im}{Im}
\DeclareMathOperator{\Li}{Li}
\DeclareMathOperator{\Ln}{Ln}
\DeclareMathOperator{\rank}{rank}
\DeclareMathOperator{\Inv}{Inv}
\DeclareMathOperator{\Fix}{Fix}
\usepackage{tkz-euclide}

\catcode`\;=\active
\newcommand{;}{\text{;}}

\usepackage[breakable,skins]{tcolorbox}
\newenvironment{mybox}{\par%
  \begin{tcolorbox}[colback=black!27,sharp corners,boxrule=-1pt,breakable]\parindent=2em
}
{\end{tcolorbox}\par}

\usepackage[numbers,square]{natbib}
\usepackage{mathdots}
\renewcommand\bibnumfmt[1]{\textbf{#1.}}
\let\olditem\itemize
\def\itemize{\olditem[left=0cm,label=\rule{1ex}{1ex}]}

\usepackage{longtable}

\usepackage{zhlineskip}
\def\Binom#1#2{
\Big(\begin{array}{@{}c@{}}
    #1\\#2
  \end{array}\Big)
}

\let\originalleft\left
\let\originalright\right
\renewcommand{\left}{\mathopen{}\mathclose\bgroup\originalleft}
\renewcommand{\right}{\aftergroup\egroup\originalright}

\definecolor{BLUE}{RGB}{0,0,255}
\let\MakeUppercase\relax
\let\olim\lim
\def\lim{\olim\limits}
\let\omax\max
\def\max{\omax\limits}
\let\omin\min
\def\min{\omin\limits}




\makeatletter
\renewcommand{\mod}[1]{\allowbreak\if@display\mkern4mu
  \else\mkern2mu\fi{\operator@font mod}\,\,#1}
\makeatother




