

\chapter*{前言}
\addcontentsline{toc} {chapter}{前言}

数学从整数开始,接下来是分数,零,以及负数. 尝试提取三次多项式的根导致了虚数单位$\ii=\sqrt{-1}$,其名字暗示着一些虚拟和可疑的东西.
除了复数之外,四元数还包括三个非交换单位. 向量源于需要追踪多个标量,例如平面或三维空间中的坐标.

矩阵在向量上表示线性函数,在科学和工程中有着极其重要的实际意义,且平凡的矩阵{--}向量方程组$Ax=b$很可能是在所有数值计算中最重要的方程. 一个偏微分方程的解可能需要一个阶数为数千或数百万的矩阵$A$,而代表网站相互依赖性的Google页面排名矩阵的顺序是几十亿. 大矩阵表示实际问题,其解决方案需要大量计算资源和创新算法. 矩阵是现代计算的必需品.

在这本独特而迷人的书中,Vasile Pop和Ovidiu Furdui避免使用大矩阵,而是将注意力集中在最简单的情况下,即$2\times2$的矩阵. 这就提出了两个问题:为什么要考虑这样的特殊情况?$2\times2$矩阵有多少有趣的数学?第二个问题有一个简洁的答案:数量惊人. 第一个问题需要更多的讨论.

由于一个$2\times2$矩阵有四项,这本书致力于研究由四元实数组构成的数学结构的性质. 例如,矩阵$A=\begin{pmatrix}
  1 & 2 \\-2 & 1
\end{pmatrix}$给出了复数$1+2\ii$的一种表示. 进一步,用$B=\begin{pmatrix}
  3 & 4 \\-4 & 3
\end{pmatrix}$来表示$3+4\ii$,那么$A+B$就表示$4+6\ii$. 类似地,$AB$表示$(1+2\ii)(3+4\ii)$,而$A^{-1}$表示$\frac1{1+2\ii}$. 因此,$2\times2$矩阵可以用来表示和处理复数,完全不需要虚数单位$\ii$.

作为$2\times2$矩阵的一个具体应用,向量微分方程$x'=Ax,A=\begin{pmatrix}
  0 & 1 \\ - \omega_n^2 & 0
\end{pmatrix}$代表了自然频率为$\omega_n$的简谐振动. 扩展矩阵$A=\begin{pmatrix}
  0 & 1 \\ -\omega_n^2 & -2\zeta\omega_n
\end{pmatrix}$包括了阻尼振动,其中$\zeta$为阻尼比. 特殊情形$A=\begin{pmatrix}
  0 & 1 \\ 0 & 0
\end{pmatrix}$拟合了没有弹簧或缓冲器的物体. 注意到$A$非零但满足$A^2=O$,$A$是幂零的. 没有一个非零的实数、复数或四元数的标量是幂零的,这个观察表明矩阵具有标量不具备的性质.

这本书的概念和范围是创新的,内容令人兴奋. 它表明一个“简单”的设置可以说明和展示丰富而有用的数学结构的美丽和错综复杂. 这本书的作者以发展和解决数学问题闻名于世,收集了大量的问题、想法和技巧. 我在期待
$n\times n$矩阵的续集. 读了这本好书,亲爱的读者,我确信
你们一定会和我一样期待.

\vspace*{1cm}
\noindent Ann Arbor,MI,USA\hfill Dennis S. Bernstein

\noindent 2016年8月

\clearpage
\chapter*{序言}\addcontentsline{toc}{chapter}{序言}


这本书是作者在过去十年里在教授线性代数课程,为学生准备大学入学考试,以及为国家和国际学生竞赛如Traian Lalescu,罗马尼亚数学竞赛,Seemous和IMC的工作成果.

本书的目标是{\kaishu 全面而详细地} 讨论与二阶矩阵理论和二维向量空间理论有关的所有重要课题以及研究平面几何、二次代数曲线、二次曲线,把微积分的初等函数推广到矩阵的函数,以及矩阵微积分在数学分析中的应用.
\begin{mybox}
  我们认为,这本书专门讨论矩阵微积分在二阶矩阵中的应用,在著作中是必要的,原因如下:
  \begin{itemize}
    \item 这也许是著作中第一本集理论、应用和涉及二阶矩阵的问题的书;
    \item 这本书的写作方式对任何有数学背景的人来说都是易懂的;
    \item 题目和问题自然扩展到$n$阶矩阵,书中的技术和思想非常新颖,可用于线性代数;
    \item 大量不同难度的问题,从简单到困难且富有挑战性,为读者学习线性代数和矩阵理论的基础知识提供了一个有价值的来源;
    \item 书中的思想和问题都是独到的,很多由作者在罗马尼亚期刊({\kaishu 数学公报B和数学公报A})和国外期刊({\kaishu 美国数学月刊、数学杂志、大学数学杂志})以及其他杂志上提出的问题将首次在著作中看到出版的曙光.
  \end{itemize}
\end{mybox}

\noindent {\hyds 这本书是写给谁的?}

你,读者,喜欢矩阵理论的人,还有那些还没有机会被介绍到矩阵的迷人世界的人. 这本书是面向那些希望学习矩阵理论的基础知识的高中生,以及想学习线性代数和矩阵理论初步的本科生,这是一个在当今世界所有的大学教授的基本课题.

我们还将这本书推荐给希望了解矩阵上某种技术应用的一年级和二年级的研究生,对于正在准备线性代数预科考试的博士生,以及任何愿意探索本书中的策略并品味涉及二阶矩阵的精彩问题的任何人.

我们也向专业人士和非专业人士推荐这本书,他们可以在一本书中找到所有你需要知道的东西,从ABC到二阶矩阵的最高级主题,以及它们与数学分析和线性代数的联系.

这本书是参加线性代数课程的学生和准备普特南、塞穆斯和国际大学生数学竞赛等数学竞赛的学生的必备品.

这本书可以被我们在高中教初等矩阵理论的同僚,或大学里教线性代数的老师,以及那些为学生准备数学竞赛的人使用.

\begin{mybox}
  \noindent {\hyds 为什么要写一本二阶矩阵的书?}
  \begin{itemize}
    \item 首先,因为在任何一门线性代数课上,学生们都与矩阵打交道,而最简单的就是二阶矩阵;
    \item 第二,由于矩阵理论中有许多漂亮的结果,参见第 \ref{chap1} 章的行列式公式,这些公式只适用于2阶矩阵,其推广到$n$维矩阵时就失去了优雅性和简洁性,以及它们证明的技巧.
    \item 第三,因为在数学中,人们首先需要理解简单的东西,而线性代数中最简单的是二阶矩阵.
    \item 第四,因为作者想把二阶矩阵及其应用的所有知识集中在一本书中,以帮助学生和教师。 尽管许多关于线性代数的优秀书籍都有一章或专门章节专门讨论二阶矩阵,但在这本书中,读者需要掌握关于矩阵理论最基本内容的所有公式.
  \end{itemize}
\end{mybox}

这本书提供了一个不寻常的问题集合,专门研究二阶矩阵,比较罕见.这些问题的难度各不相同,从最简单的计算矩阵的$n$次方到最高级的如第 \ref{chap4} 章中的问题. 这本书中的大部分问题都是新的和原创的,而且是第一次出版. 其他的灵感来自西方文学中没有的几本书 \cite{47,48,50,51}. 一些问题的另一个重要灵感来源是著名的罗马尼亚期刊{\kaishu 数学公报},罗马尼亚最古老的数学出版物,第一期出版于1895年,以及世界上第一本有问题专栏的数学杂志。我们并不主张本卷所包含的所有问题的独创性,而且我们知道有些习题和结论是已知的或非常陈旧的.

我们详细地解决了大部分问题,但也有一些练习没有解决. 这是因为我们想激励读者不要遵循我们解决问题的技巧,而是开发自己的解题方法. 有几个问题是挑战性问题,这些新的问题和新的解决方案可以激发读者的创造性发现和证明.

\noindent{\hyds 这本书的内容}

这本书有六章,还有两个附录.

在第 \ref{chap1} 章中,我们回顾了基本结论和定义,并给出了特殊矩阵的结构,如幂等矩阵、幂零矩阵、对合矩阵、反对合矩阵和正交矩阵. 我们还讨论了矩阵的中心化子. 这一章包含行列式计算的特殊问题,以及一些经典代数结构的练习,如群、环、矩阵域及其性质.

第 \ref{chap2} 章讨论著名的Cayley--Hamilton定理,以及它的相关问题,Jordan标准形,特殊矩阵的实标准形和有理标准形. 此外,本章还包含一个名为quickies的部分,它是关于一些具有难以预料的简洁解决方案的问题.

第 \ref{chap3} 章给出了矩阵的$n$次方的计算公式,研究了由线性和等交比的递归关系定义的序列,求解了二项矩阵方程,并回顾了著名的Pell方程. 本章有一节专门讨论二项式方程$X^n=aI_2,\,a\in\MR^\ast$,我们认为这是第一次出现在著作中. 这些问题是新颖和独创性的,请参阅有关Vi\'ete公式的二次矩阵方程的问题,并挑战读者探索整个章节中讨论的工具.

第 \ref{chap4} 章,{\hyds 这本书的精华},是矩阵理论和数学的混合体分析. 本章包含矩阵的数列和级数,初等的矩阵函数,并在著作中(我们相信是)首次引入矩阵的{\hyds Riemann--zeta函数} 和{\hyds Gamma函数}. 各种各样的问题,从指数函数的计算,微分方程组求解,到一重或二重矩阵积分,以及{\hyds Frullani矩阵积分}的计算.

第 \ref{chap5} 章,在著作中{\hyds 独树一帜},是关于平面的特殊线性应用,如投影和反射及其基本性质的研究. 本章中的大部分问题都是线性代数的瑰宝,都是第一次出现在著作中.

在第 \ref{chap6} 章中,我们使用Jordan标准形将二次代数曲线(圆锥曲线)约化为它们的标准形. 这一章中的问题既不是标准的,也不是已知的:它们的范围从约化圆锥曲线为它的标准形式,到极值问题的研究,甚至是椭圆域上二重积分的计算.

附录 \ref{chapA} 包含了一系列的主题,从线性代数,如矩阵的指数函数和三角函数的计算,到经典分析的主题,如非标准级数的计算和两个新的分析中的Frullani积分的讨论.

在附录 \ref{chapB},但愿是{\kaishu 新颖的线性代数},我们解决三角矩阵方程,并邀请读者进一步探讨这些主题.

{\kaishu 这本书旨在吸引新手,迷惑专家,激发所有人的想象力}. 它包含了矩阵理论中一系列不寻常的问题,我们认为这些问题很{\kaishu 精彩}. 问题是否很精彩(我们相信是的!),当然这是由你们读者来决定的. 我们希望你们会喜欢这些问题和理论.

\begin{mybox}
{\hyds 声明}\parindent=2em

  我们对来自奥尔阿斯蒂的Ioan \c Serdean教授表示感谢,他为本书提供了参考资料.

  非常感谢Alina S\^int\u am\u arian,她阅读了本书的大部分内容,并帮助我们改进了本材料的呈现方式. 她还提供了参考资料,并协助第二作者解决排版过程中出现的各种问题. Alina还{\kaishu 负责}使用GeoGebra软件包绘制本书中的所有图形\footnote{译者将会用 {\color{red}TikZ} 把插图全部重新绘制.},再次{\kaishu 感谢}她在编写本书过程中提出的宝贵意见.
\end{mybox}

谢谢大家,享受精彩的$2\times2$矩阵!!!
\vspace*{1cm}

\noindent Cluj-Napoca,\hfill Romania Vasile Pop

\noindent2016年6月 \hfill Ovidiu Furdui










\clearpage
